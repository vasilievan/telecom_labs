\documentclass[11pt]{article}

    \usepackage[breakable]{tcolorbox}
    \usepackage{parskip} % Stop auto-indenting (to mimic markdown behaviour)
     \usepackage[utf8]{inputenc}
\usepackage[russian]{babel}

    % Basic figure setup, for now with no caption control since it's done
    % automatically by Pandoc (which extracts ![](path) syntax from Markdown).
    \usepackage{graphicx}
    % Maintain compatibility with old templates. Remove in nbconvert 6.0
    \let\Oldincludegraphics\includegraphics
    % Ensure that by default, figures have no caption (until we provide a
    % proper Figure object with a Caption API and a way to capture that
    % in the conversion process - todo).
    \usepackage{caption}
    \DeclareCaptionFormat{nocaption}{}
    \captionsetup{format=nocaption,aboveskip=0pt,belowskip=0pt}

    \usepackage{float}
    \floatplacement{figure}{H} % forces figures to be placed at the correct location
    \usepackage{xcolor} % Allow colors to be defined
    \usepackage{enumerate} % Needed for markdown enumerations to work
    \usepackage{geometry} % Used to adjust the document margins
    \usepackage{amsmath} % Equations
    \usepackage{amssymb} % Equations
    \usepackage{textcomp} % defines textquotesingle
    % Hack from http://tex.stackexchange.com/a/47451/13684:
    \AtBeginDocument{%
        \def\PYZsq{\textquotesingle}% Upright quotes in Pygmentized code
    }
    \usepackage{upquote} % Upright quotes for verbatim code
    \usepackage{eurosym} % defines \euro

    \usepackage{iftex}
    \ifPDFTeX
        \usepackage[T1]{fontenc}
        \IfFileExists{alphabeta.sty}{
              \usepackage{alphabeta}
          }{
              \usepackage[mathletters]{ucs}
              \usepackage[utf8x]{inputenc}
          }
    \else
        \usepackage{fontspec}
        \usepackage{unicode-math}
    \fi

    \usepackage{fancyvrb} % verbatim replacement that allows latex
    \usepackage{grffile} % extends the file name processing of package graphics
                         % to support a larger range
    \makeatletter % fix for old versions of grffile with XeLaTeX
    \@ifpackagelater{grffile}{2019/11/01}
    {
      % Do nothing on new versions
    }
    {
      \def\Gread@@xetex#1{%
        \IfFileExists{"\Gin@base".bb}%
        {\Gread@eps{\Gin@base.bb}}%
        {\Gread@@xetex@aux#1}%
      }
    }
    \makeatother
    \usepackage[Export]{adjustbox} % Used to constrain images to a maximum size
    \adjustboxset{max size={0.9\linewidth}{0.9\paperheight}}

    % The hyperref package gives us a pdf with properly built
    % internal navigation ('pdf bookmarks' for the table of contents,
    % internal cross-reference links, web links for URLs, etc.)
    \usepackage{hyperref}
    % The default LaTeX title has an obnoxious amount of whitespace. By default,
    % titling removes some of it. It also provides customization options.
    \usepackage{titling}
    \usepackage{longtable} % longtable support required by pandoc >1.10
    \usepackage{booktabs}  % table support for pandoc > 1.12.2
    \usepackage{array}     % table support for pandoc >= 2.11.3
    \usepackage{calc}      % table minipage width calculation for pandoc >= 2.11.1
    \usepackage[inline]{enumitem} % IRkernel/repr support (it uses the enumerate* environment)
    \usepackage[normalem]{ulem} % ulem is needed to support strikethroughs (\sout)
                                % normalem makes italics be italics, not underlines
    \usepackage{mathrsfs}
    

    
    % Colors for the hyperref package
    \definecolor{urlcolor}{rgb}{0,.145,.698}
    \definecolor{linkcolor}{rgb}{.71,0.21,0.01}
    \definecolor{citecolor}{rgb}{.12,.54,.11}

    % ANSI colors
    \definecolor{ansi-black}{HTML}{3E424D}
    \definecolor{ansi-black-intense}{HTML}{282C36}
    \definecolor{ansi-red}{HTML}{E75C58}
    \definecolor{ansi-red-intense}{HTML}{B22B31}
    \definecolor{ansi-green}{HTML}{00A250}
    \definecolor{ansi-green-intense}{HTML}{007427}
    \definecolor{ansi-yellow}{HTML}{DDB62B}
    \definecolor{ansi-yellow-intense}{HTML}{B27D12}
    \definecolor{ansi-blue}{HTML}{208FFB}
    \definecolor{ansi-blue-intense}{HTML}{0065CA}
    \definecolor{ansi-magenta}{HTML}{D160C4}
    \definecolor{ansi-magenta-intense}{HTML}{A03196}
    \definecolor{ansi-cyan}{HTML}{60C6C8}
    \definecolor{ansi-cyan-intense}{HTML}{258F8F}
    \definecolor{ansi-white}{HTML}{C5C1B4}
    \definecolor{ansi-white-intense}{HTML}{A1A6B2}
    \definecolor{ansi-default-inverse-fg}{HTML}{FFFFFF}
    \definecolor{ansi-default-inverse-bg}{HTML}{000000}

    % common color for the border for error outputs.
    \definecolor{outerrorbackground}{HTML}{FFDFDF}

    % commands and environments needed by pandoc snippets
    % extracted from the output of `pandoc -s`
    \providecommand{\tightlist}{%
      \setlength{\itemsep}{0pt}\setlength{\parskip}{0pt}}
    \DefineVerbatimEnvironment{Highlighting}{Verbatim}{commandchars=\\\{\}}
    % Add ',fontsize=\small' for more characters per line
    \newenvironment{Shaded}{}{}
    \newcommand{\KeywordTok}[1]{\textcolor[rgb]{0.00,0.44,0.13}{\textbf{{#1}}}}
    \newcommand{\DataTypeTok}[1]{\textcolor[rgb]{0.56,0.13,0.00}{{#1}}}
    \newcommand{\DecValTok}[1]{\textcolor[rgb]{0.25,0.63,0.44}{{#1}}}
    \newcommand{\BaseNTok}[1]{\textcolor[rgb]{0.25,0.63,0.44}{{#1}}}
    \newcommand{\FloatTok}[1]{\textcolor[rgb]{0.25,0.63,0.44}{{#1}}}
    \newcommand{\CharTok}[1]{\textcolor[rgb]{0.25,0.44,0.63}{{#1}}}
    \newcommand{\StringTok}[1]{\textcolor[rgb]{0.25,0.44,0.63}{{#1}}}
    \newcommand{\CommentTok}[1]{\textcolor[rgb]{0.38,0.63,0.69}{\textit{{#1}}}}
    \newcommand{\OtherTok}[1]{\textcolor[rgb]{0.00,0.44,0.13}{{#1}}}
    \newcommand{\AlertTok}[1]{\textcolor[rgb]{1.00,0.00,0.00}{\textbf{{#1}}}}
    \newcommand{\FunctionTok}[1]{\textcolor[rgb]{0.02,0.16,0.49}{{#1}}}
    \newcommand{\RegionMarkerTok}[1]{{#1}}
    \newcommand{\ErrorTok}[1]{\textcolor[rgb]{1.00,0.00,0.00}{\textbf{{#1}}}}
    \newcommand{\NormalTok}[1]{{#1}}

    % Additional commands for more recent versions of Pandoc
    \newcommand{\ConstantTok}[1]{\textcolor[rgb]{0.53,0.00,0.00}{{#1}}}
    \newcommand{\SpecialCharTok}[1]{\textcolor[rgb]{0.25,0.44,0.63}{{#1}}}
    \newcommand{\VerbatimStringTok}[1]{\textcolor[rgb]{0.25,0.44,0.63}{{#1}}}
    \newcommand{\SpecialStringTok}[1]{\textcolor[rgb]{0.73,0.40,0.53}{{#1}}}
    \newcommand{\ImportTok}[1]{{#1}}
    \newcommand{\DocumentationTok}[1]{\textcolor[rgb]{0.73,0.13,0.13}{\textit{{#1}}}}
    \newcommand{\AnnotationTok}[1]{\textcolor[rgb]{0.38,0.63,0.69}{\textbf{\textit{{#1}}}}}
    \newcommand{\CommentVarTok}[1]{\textcolor[rgb]{0.38,0.63,0.69}{\textbf{\textit{{#1}}}}}
    \newcommand{\VariableTok}[1]{\textcolor[rgb]{0.10,0.09,0.49}{{#1}}}
    \newcommand{\ControlFlowTok}[1]{\textcolor[rgb]{0.00,0.44,0.13}{\textbf{{#1}}}}
    \newcommand{\OperatorTok}[1]{\textcolor[rgb]{0.40,0.40,0.40}{{#1}}}
    \newcommand{\BuiltInTok}[1]{{#1}}
    \newcommand{\ExtensionTok}[1]{{#1}}
    \newcommand{\PreprocessorTok}[1]{\textcolor[rgb]{0.74,0.48,0.00}{{#1}}}
    \newcommand{\AttributeTok}[1]{\textcolor[rgb]{0.49,0.56,0.16}{{#1}}}
    \newcommand{\InformationTok}[1]{\textcolor[rgb]{0.38,0.63,0.69}{\textbf{\textit{{#1}}}}}
    \newcommand{\WarningTok}[1]{\textcolor[rgb]{0.38,0.63,0.69}{\textbf{\textit{{#1}}}}}


    % Define a nice break command that doesn't care if a line doesn't already
    % exist.
    \def\br{\hspace*{\fill} \\* }
    % Math Jax compatibility definitions
    \def\gt{>}
    \def\lt{<}
    \let\Oldtex\TeX
    \let\Oldlatex\LaTeX
    \renewcommand{\TeX}{\textrm{\Oldtex}}
    \renewcommand{\LaTeX}{\textrm{\Oldlatex}}
    % Document parameters
    % Document title
    \title{Vasiliev\_lab4}
    
    
    
    
    
% Pygments definitions
\makeatletter
\def\PY@reset{\let\PY@it=\relax \let\PY@bf=\relax%
    \let\PY@ul=\relax \let\PY@tc=\relax%
    \let\PY@bc=\relax \let\PY@ff=\relax}
\def\PY@tok#1{\csname PY@tok@#1\endcsname}
\def\PY@toks#1+{\ifx\relax#1\empty\else%
    \PY@tok{#1}\expandafter\PY@toks\fi}
\def\PY@do#1{\PY@bc{\PY@tc{\PY@ul{%
    \PY@it{\PY@bf{\PY@ff{#1}}}}}}}
\def\PY#1#2{\PY@reset\PY@toks#1+\relax+\PY@do{#2}}

\@namedef{PY@tok@w}{\def\PY@tc##1{\textcolor[rgb]{0.73,0.73,0.73}{##1}}}
\@namedef{PY@tok@c}{\let\PY@it=\textit\def\PY@tc##1{\textcolor[rgb]{0.24,0.48,0.48}{##1}}}
\@namedef{PY@tok@cp}{\def\PY@tc##1{\textcolor[rgb]{0.61,0.40,0.00}{##1}}}
\@namedef{PY@tok@k}{\let\PY@bf=\textbf\def\PY@tc##1{\textcolor[rgb]{0.00,0.50,0.00}{##1}}}
\@namedef{PY@tok@kp}{\def\PY@tc##1{\textcolor[rgb]{0.00,0.50,0.00}{##1}}}
\@namedef{PY@tok@kt}{\def\PY@tc##1{\textcolor[rgb]{0.69,0.00,0.25}{##1}}}
\@namedef{PY@tok@o}{\def\PY@tc##1{\textcolor[rgb]{0.40,0.40,0.40}{##1}}}
\@namedef{PY@tok@ow}{\let\PY@bf=\textbf\def\PY@tc##1{\textcolor[rgb]{0.67,0.13,1.00}{##1}}}
\@namedef{PY@tok@nb}{\def\PY@tc##1{\textcolor[rgb]{0.00,0.50,0.00}{##1}}}
\@namedef{PY@tok@nf}{\def\PY@tc##1{\textcolor[rgb]{0.00,0.00,1.00}{##1}}}
\@namedef{PY@tok@nc}{\let\PY@bf=\textbf\def\PY@tc##1{\textcolor[rgb]{0.00,0.00,1.00}{##1}}}
\@namedef{PY@tok@nn}{\let\PY@bf=\textbf\def\PY@tc##1{\textcolor[rgb]{0.00,0.00,1.00}{##1}}}
\@namedef{PY@tok@ne}{\let\PY@bf=\textbf\def\PY@tc##1{\textcolor[rgb]{0.80,0.25,0.22}{##1}}}
\@namedef{PY@tok@nv}{\def\PY@tc##1{\textcolor[rgb]{0.10,0.09,0.49}{##1}}}
\@namedef{PY@tok@no}{\def\PY@tc##1{\textcolor[rgb]{0.53,0.00,0.00}{##1}}}
\@namedef{PY@tok@nl}{\def\PY@tc##1{\textcolor[rgb]{0.46,0.46,0.00}{##1}}}
\@namedef{PY@tok@ni}{\let\PY@bf=\textbf\def\PY@tc##1{\textcolor[rgb]{0.44,0.44,0.44}{##1}}}
\@namedef{PY@tok@na}{\def\PY@tc##1{\textcolor[rgb]{0.41,0.47,0.13}{##1}}}
\@namedef{PY@tok@nt}{\let\PY@bf=\textbf\def\PY@tc##1{\textcolor[rgb]{0.00,0.50,0.00}{##1}}}
\@namedef{PY@tok@nd}{\def\PY@tc##1{\textcolor[rgb]{0.67,0.13,1.00}{##1}}}
\@namedef{PY@tok@s}{\def\PY@tc##1{\textcolor[rgb]{0.73,0.13,0.13}{##1}}}
\@namedef{PY@tok@sd}{\let\PY@it=\textit\def\PY@tc##1{\textcolor[rgb]{0.73,0.13,0.13}{##1}}}
\@namedef{PY@tok@si}{\let\PY@bf=\textbf\def\PY@tc##1{\textcolor[rgb]{0.64,0.35,0.47}{##1}}}
\@namedef{PY@tok@se}{\let\PY@bf=\textbf\def\PY@tc##1{\textcolor[rgb]{0.67,0.36,0.12}{##1}}}
\@namedef{PY@tok@sr}{\def\PY@tc##1{\textcolor[rgb]{0.64,0.35,0.47}{##1}}}
\@namedef{PY@tok@ss}{\def\PY@tc##1{\textcolor[rgb]{0.10,0.09,0.49}{##1}}}
\@namedef{PY@tok@sx}{\def\PY@tc##1{\textcolor[rgb]{0.00,0.50,0.00}{##1}}}
\@namedef{PY@tok@m}{\def\PY@tc##1{\textcolor[rgb]{0.40,0.40,0.40}{##1}}}
\@namedef{PY@tok@gh}{\let\PY@bf=\textbf\def\PY@tc##1{\textcolor[rgb]{0.00,0.00,0.50}{##1}}}
\@namedef{PY@tok@gu}{\let\PY@bf=\textbf\def\PY@tc##1{\textcolor[rgb]{0.50,0.00,0.50}{##1}}}
\@namedef{PY@tok@gd}{\def\PY@tc##1{\textcolor[rgb]{0.63,0.00,0.00}{##1}}}
\@namedef{PY@tok@gi}{\def\PY@tc##1{\textcolor[rgb]{0.00,0.52,0.00}{##1}}}
\@namedef{PY@tok@gr}{\def\PY@tc##1{\textcolor[rgb]{0.89,0.00,0.00}{##1}}}
\@namedef{PY@tok@ge}{\let\PY@it=\textit}
\@namedef{PY@tok@gs}{\let\PY@bf=\textbf}
\@namedef{PY@tok@gp}{\let\PY@bf=\textbf\def\PY@tc##1{\textcolor[rgb]{0.00,0.00,0.50}{##1}}}
\@namedef{PY@tok@go}{\def\PY@tc##1{\textcolor[rgb]{0.44,0.44,0.44}{##1}}}
\@namedef{PY@tok@gt}{\def\PY@tc##1{\textcolor[rgb]{0.00,0.27,0.87}{##1}}}
\@namedef{PY@tok@err}{\def\PY@bc##1{{\setlength{\fboxsep}{\string -\fboxrule}\fcolorbox[rgb]{1.00,0.00,0.00}{1,1,1}{\strut ##1}}}}
\@namedef{PY@tok@kc}{\let\PY@bf=\textbf\def\PY@tc##1{\textcolor[rgb]{0.00,0.50,0.00}{##1}}}
\@namedef{PY@tok@kd}{\let\PY@bf=\textbf\def\PY@tc##1{\textcolor[rgb]{0.00,0.50,0.00}{##1}}}
\@namedef{PY@tok@kn}{\let\PY@bf=\textbf\def\PY@tc##1{\textcolor[rgb]{0.00,0.50,0.00}{##1}}}
\@namedef{PY@tok@kr}{\let\PY@bf=\textbf\def\PY@tc##1{\textcolor[rgb]{0.00,0.50,0.00}{##1}}}
\@namedef{PY@tok@bp}{\def\PY@tc##1{\textcolor[rgb]{0.00,0.50,0.00}{##1}}}
\@namedef{PY@tok@fm}{\def\PY@tc##1{\textcolor[rgb]{0.00,0.00,1.00}{##1}}}
\@namedef{PY@tok@vc}{\def\PY@tc##1{\textcolor[rgb]{0.10,0.09,0.49}{##1}}}
\@namedef{PY@tok@vg}{\def\PY@tc##1{\textcolor[rgb]{0.10,0.09,0.49}{##1}}}
\@namedef{PY@tok@vi}{\def\PY@tc##1{\textcolor[rgb]{0.10,0.09,0.49}{##1}}}
\@namedef{PY@tok@vm}{\def\PY@tc##1{\textcolor[rgb]{0.10,0.09,0.49}{##1}}}
\@namedef{PY@tok@sa}{\def\PY@tc##1{\textcolor[rgb]{0.73,0.13,0.13}{##1}}}
\@namedef{PY@tok@sb}{\def\PY@tc##1{\textcolor[rgb]{0.73,0.13,0.13}{##1}}}
\@namedef{PY@tok@sc}{\def\PY@tc##1{\textcolor[rgb]{0.73,0.13,0.13}{##1}}}
\@namedef{PY@tok@dl}{\def\PY@tc##1{\textcolor[rgb]{0.73,0.13,0.13}{##1}}}
\@namedef{PY@tok@s2}{\def\PY@tc##1{\textcolor[rgb]{0.73,0.13,0.13}{##1}}}
\@namedef{PY@tok@sh}{\def\PY@tc##1{\textcolor[rgb]{0.73,0.13,0.13}{##1}}}
\@namedef{PY@tok@s1}{\def\PY@tc##1{\textcolor[rgb]{0.73,0.13,0.13}{##1}}}
\@namedef{PY@tok@mb}{\def\PY@tc##1{\textcolor[rgb]{0.40,0.40,0.40}{##1}}}
\@namedef{PY@tok@mf}{\def\PY@tc##1{\textcolor[rgb]{0.40,0.40,0.40}{##1}}}
\@namedef{PY@tok@mh}{\def\PY@tc##1{\textcolor[rgb]{0.40,0.40,0.40}{##1}}}
\@namedef{PY@tok@mi}{\def\PY@tc##1{\textcolor[rgb]{0.40,0.40,0.40}{##1}}}
\@namedef{PY@tok@il}{\def\PY@tc##1{\textcolor[rgb]{0.40,0.40,0.40}{##1}}}
\@namedef{PY@tok@mo}{\def\PY@tc##1{\textcolor[rgb]{0.40,0.40,0.40}{##1}}}
\@namedef{PY@tok@ch}{\let\PY@it=\textit\def\PY@tc##1{\textcolor[rgb]{0.24,0.48,0.48}{##1}}}
\@namedef{PY@tok@cm}{\let\PY@it=\textit\def\PY@tc##1{\textcolor[rgb]{0.24,0.48,0.48}{##1}}}
\@namedef{PY@tok@cpf}{\let\PY@it=\textit\def\PY@tc##1{\textcolor[rgb]{0.24,0.48,0.48}{##1}}}
\@namedef{PY@tok@c1}{\let\PY@it=\textit\def\PY@tc##1{\textcolor[rgb]{0.24,0.48,0.48}{##1}}}
\@namedef{PY@tok@cs}{\let\PY@it=\textit\def\PY@tc##1{\textcolor[rgb]{0.24,0.48,0.48}{##1}}}

\def\PYZbs{\char`\\}
\def\PYZus{\char`\_}
\def\PYZob{\char`\{}
\def\PYZcb{\char`\}}
\def\PYZca{\char`\^}
\def\PYZam{\char`\&}
\def\PYZlt{\char`\<}
\def\PYZgt{\char`\>}
\def\PYZsh{\char`\#}
\def\PYZpc{\char`\%}
\def\PYZdl{\char`\$}
\def\PYZhy{\char`\-}
\def\PYZsq{\char`\'}
\def\PYZdq{\char`\"}
\def\PYZti{\char`\~}
% for compatibility with earlier versions
\def\PYZat{@}
\def\PYZlb{[}
\def\PYZrb{]}
\makeatother


    % For linebreaks inside Verbatim environment from package fancyvrb.
    \makeatletter
        \newbox\Wrappedcontinuationbox
        \newbox\Wrappedvisiblespacebox
        \newcommand*\Wrappedvisiblespace {\textcolor{red}{\textvisiblespace}}
        \newcommand*\Wrappedcontinuationsymbol {\textcolor{red}{\llap{\tiny$\m@th\hookrightarrow$}}}
        \newcommand*\Wrappedcontinuationindent {3ex }
        \newcommand*\Wrappedafterbreak {\kern\Wrappedcontinuationindent\copy\Wrappedcontinuationbox}
        % Take advantage of the already applied Pygments mark-up to insert
        % potential linebreaks for TeX processing.
        %        {, <, #, %, $, ' and ": go to next line.
        %        _, }, ^, &, >, - and ~: stay at end of broken line.
        % Use of \textquotesingle for straight quote.
        \newcommand*\Wrappedbreaksatspecials {%
            \def\PYGZus{\discretionary{\char`\_}{\Wrappedafterbreak}{\char`\_}}%
            \def\PYGZob{\discretionary{}{\Wrappedafterbreak\char`\{}{\char`\{}}%
            \def\PYGZcb{\discretionary{\char`\}}{\Wrappedafterbreak}{\char`\}}}%
            \def\PYGZca{\discretionary{\char`\^}{\Wrappedafterbreak}{\char`\^}}%
            \def\PYGZam{\discretionary{\char`\&}{\Wrappedafterbreak}{\char`\&}}%
            \def\PYGZlt{\discretionary{}{\Wrappedafterbreak\char`\<}{\char`\<}}%
            \def\PYGZgt{\discretionary{\char`\>}{\Wrappedafterbreak}{\char`\>}}%
            \def\PYGZsh{\discretionary{}{\Wrappedafterbreak\char`\#}{\char`\#}}%
            \def\PYGZpc{\discretionary{}{\Wrappedafterbreak\char`\%}{\char`\%}}%
            \def\PYGZdl{\discretionary{}{\Wrappedafterbreak\char`\$}{\char`\$}}%
            \def\PYGZhy{\discretionary{\char`\-}{\Wrappedafterbreak}{\char`\-}}%
            \def\PYGZsq{\discretionary{}{\Wrappedafterbreak\textquotesingle}{\textquotesingle}}%
            \def\PYGZdq{\discretionary{}{\Wrappedafterbreak\char`\"}{\char`\"}}%
            \def\PYGZti{\discretionary{\char`\~}{\Wrappedafterbreak}{\char`\~}}%
        }
        % Some characters . , ; ? ! / are not pygmentized.
        % This macro makes them "active" and they will insert potential linebreaks
        \newcommand*\Wrappedbreaksatpunct {%
            \lccode`\~`\.\lowercase{\def~}{\discretionary{\hbox{\char`\.}}{\Wrappedafterbreak}{\hbox{\char`\.}}}%
            \lccode`\~`\,\lowercase{\def~}{\discretionary{\hbox{\char`\,}}{\Wrappedafterbreak}{\hbox{\char`\,}}}%
            \lccode`\~`\;\lowercase{\def~}{\discretionary{\hbox{\char`\;}}{\Wrappedafterbreak}{\hbox{\char`\;}}}%
            \lccode`\~`\:\lowercase{\def~}{\discretionary{\hbox{\char`\:}}{\Wrappedafterbreak}{\hbox{\char`\:}}}%
            \lccode`\~`\?\lowercase{\def~}{\discretionary{\hbox{\char`\?}}{\Wrappedafterbreak}{\hbox{\char`\?}}}%
            \lccode`\~`\!\lowercase{\def~}{\discretionary{\hbox{\char`\!}}{\Wrappedafterbreak}{\hbox{\char`\!}}}%
            \lccode`\~`\/\lowercase{\def~}{\discretionary{\hbox{\char`\/}}{\Wrappedafterbreak}{\hbox{\char`\/}}}%
            \catcode`\.\active
            \catcode`\,\active
            \catcode`\;\active
            \catcode`\:\active
            \catcode`\?\active
            \catcode`\!\active
            \catcode`\/\active
            \lccode`\~`\~
        }
    \makeatother

    \let\OriginalVerbatim=\Verbatim
    \makeatletter
    \renewcommand{\Verbatim}[1][1]{%
        %\parskip\z@skip
        \sbox\Wrappedcontinuationbox {\Wrappedcontinuationsymbol}%
        \sbox\Wrappedvisiblespacebox {\FV@SetupFont\Wrappedvisiblespace}%
        \def\FancyVerbFormatLine ##1{\hsize\linewidth
            \vtop{\raggedright\hyphenpenalty\z@\exhyphenpenalty\z@
                \doublehyphendemerits\z@\finalhyphendemerits\z@
                \strut ##1\strut}%
        }%
        % If the linebreak is at a space, the latter will be displayed as visible
        % space at end of first line, and a continuation symbol starts next line.
        % Stretch/shrink are however usually zero for typewriter font.
        \def\FV@Space {%
            \nobreak\hskip\z@ plus\fontdimen3\font minus\fontdimen4\font
            \discretionary{\copy\Wrappedvisiblespacebox}{\Wrappedafterbreak}
            {\kern\fontdimen2\font}%
        }%

        % Allow breaks at special characters using \PYG... macros.
        \Wrappedbreaksatspecials
        % Breaks at punctuation characters . , ; ? ! and / need catcode=\active
        \OriginalVerbatim[#1,codes*=\Wrappedbreaksatpunct]%
    }
    \makeatother

    % Exact colors from NB
    \definecolor{incolor}{HTML}{303F9F}
    \definecolor{outcolor}{HTML}{D84315}
    \definecolor{cellborder}{HTML}{CFCFCF}
    \definecolor{cellbackground}{HTML}{F7F7F7}

    % prompt
    \makeatletter
    \newcommand{\boxspacing}{\kern\kvtcb@left@rule\kern\kvtcb@boxsep}
    \makeatother
    \newcommand{\prompt}[4]{
        {\ttfamily\llap{{\color{#2}[#3]:\hspace{3pt}#4}}\vspace{-\baselineskip}}
    }
    

    
    % Prevent overflowing lines due to hard-to-break entities
    \sloppy
    % Setup hyperref package
    \hypersetup{
      breaklinks=true,  % so long urls are correctly broken across lines
      colorlinks=true,
      urlcolor=urlcolor,
      linkcolor=linkcolor,
      citecolor=citecolor,
      }
    % Slightly bigger margins than the latex defaults
    
    \geometry{verbose,tmargin=1in,bmargin=1in,lmargin=1in,rmargin=1in}
    
    

\begin{document}
    
    \maketitle
    
    

    
    

    \begin{tcolorbox}[breakable, size=fbox, boxrule=1pt, pad at break*=1mm,colback=cellbackground, colframe=cellborder]
\prompt{In}{incolor}{ }{\boxspacing}
\begin{Verbatim}[commandchars=\\\{\}]
\PY{c+c1}{\PYZsh{} Get thinkdsp.py}

\PY{k+kn}{import} \PY{n+nn}{os}

\PY{k}{if} \PY{o+ow}{not} \PY{n}{os}\PY{o}{.}\PY{n}{path}\PY{o}{.}\PY{n}{exists}\PY{p}{(}\PY{l+s+s1}{\PYZsq{}}\PY{l+s+s1}{thinkdsp.py}\PY{l+s+s1}{\PYZsq{}}\PY{p}{)}\PY{p}{:}
    \PY{err}{!}\PY{n}{wget} \PY{n}{https}\PY{p}{:}\PY{o}{/}\PY{o}{/}\PY{n}{github}\PY{o}{.}\PY{n}{com}\PY{o}{/}\PY{n}{AllenDowney}\PY{o}{/}\PY{n}{ThinkDSP}\PY{o}{/}\PY{n}{raw}\PY{o}{/}\PY{n}{master}\PY{o}{/}\PY{n}{code}\PY{o}{/}\PY{n}{thinkdsp}\PY{o}{.}\PY{n}{py}
\end{Verbatim}
\end{tcolorbox}

    \begin{Verbatim}[commandchars=\\\{\}]
--2022-04-08 12:02:56--
https://github.com/AllenDowney/ThinkDSP/raw/master/code/thinkdsp.py
Resolving github.com (github.com){\ldots} 140.82.112.3
Connecting to github.com (github.com)|140.82.112.3|:443{\ldots} connected.
HTTP request sent, awaiting response{\ldots} 302 Found
Location:
https://raw.githubusercontent.com/AllenDowney/ThinkDSP/master/code/thinkdsp.py
[following]
--2022-04-08 12:02:56--
https://raw.githubusercontent.com/AllenDowney/ThinkDSP/master/code/thinkdsp.py
Resolving raw.githubusercontent.com (raw.githubusercontent.com){\ldots}
185.199.108.133, 185.199.109.133, 185.199.110.133, {\ldots}
Connecting to raw.githubusercontent.com
(raw.githubusercontent.com)|185.199.108.133|:443{\ldots} connected.
HTTP request sent, awaiting response{\ldots} 200 OK
Length: 48687 (48K) [text/plain]
Saving to: ‘thinkdsp.py’

thinkdsp.py         100\%[===================>]  47.55K  --.-KB/s    in 0.009s

2022-04-08 12:02:56 (4.91 MB/s) - ‘thinkdsp.py’ saved [48687/48687]

    \end{Verbatim}

    \begin{tcolorbox}[breakable, size=fbox, boxrule=1pt, pad at break*=1mm,colback=cellbackground, colframe=cellborder]
\prompt{In}{incolor}{ }{\boxspacing}
\begin{Verbatim}[commandchars=\\\{\}]
\PY{k+kn}{import} \PY{n+nn}{numpy} \PY{k}{as} \PY{n+nn}{np}
\PY{k+kn}{import} \PY{n+nn}{matplotlib}\PY{n+nn}{.}\PY{n+nn}{pyplot} \PY{k}{as} \PY{n+nn}{plt}

\PY{k+kn}{from} \PY{n+nn}{thinkdsp} \PY{k+kn}{import} \PY{n}{decorate}
\end{Verbatim}
\end{tcolorbox}

    \hypertarget{ux443ux43fux440ux430ux436ux43dux435ux43dux438ux435-2}{%
\subsection{Упражнение
2}\label{ux443ux43fux440ux430ux436ux43dux435ux43dux438ux435-2}}

    \begin{tcolorbox}[breakable, size=fbox, boxrule=1pt, pad at break*=1mm,colback=cellbackground, colframe=cellborder]
\prompt{In}{incolor}{ }{\boxspacing}
\begin{Verbatim}[commandchars=\\\{\}]
\PY{k}{if} \PY{o+ow}{not} \PY{n}{os}\PY{o}{.}\PY{n}{path}\PY{o}{.}\PY{n}{exists}\PY{p}{(}\PY{l+s+s1}{\PYZsq{}}\PY{l+s+s1}{29530\PYZus{}\PYZus{}eliasheuninck\PYZus{}\PYZus{}wind1.wav}\PY{l+s+s1}{\PYZsq{}}\PY{p}{)}\PY{p}{:}
    \PY{err}{!}\PY{n}{wget} \PY{n}{https}\PY{p}{:}\PY{o}{/}\PY{o}{/}\PY{n}{github}\PY{o}{.}\PY{n}{com}\PY{o}{/}\PY{n}{vasilievan}\PY{o}{/}\PY{n}{telecom\PYZus{}labs}\PY{o}{/}\PY{n}{raw}\PY{o}{/}\PY{n}{main}\PY{o}{/}\PY{l+m+mi}{29530}\PY{n}{\PYZus{}\PYZus{}eliasheuninck\PYZus{}\PYZus{}wind1}\PY{o}{.}\PY{n}{wav}
\end{Verbatim}
\end{tcolorbox}

    \begin{tcolorbox}[breakable, size=fbox, boxrule=1pt, pad at break*=1mm,colback=cellbackground, colframe=cellborder]
\prompt{In}{incolor}{ }{\boxspacing}
\begin{Verbatim}[commandchars=\\\{\}]
\PY{k+kn}{from} \PY{n+nn}{thinkdsp} \PY{k+kn}{import} \PY{n}{read\PYZus{}wave}

\PY{n}{wave} \PY{o}{=} \PY{n}{read\PYZus{}wave}\PY{p}{(}\PY{l+s+s1}{\PYZsq{}}\PY{l+s+s1}{29530\PYZus{}\PYZus{}eliasheuninck\PYZus{}\PYZus{}wind1.wav}\PY{l+s+s1}{\PYZsq{}}\PY{p}{)}
\PY{n}{wave}\PY{o}{.}\PY{n}{make\PYZus{}audio}\PY{p}{(}\PY{p}{)}
\end{Verbatim}
\end{tcolorbox}

            \begin{tcolorbox}[breakable, size=fbox, boxrule=.5pt, pad at break*=1mm, opacityfill=0]
\prompt{Out}{outcolor}{ }{\boxspacing}
\begin{Verbatim}[commandchars=\\\{\}]
<IPython.lib.display.Audio object>
\end{Verbatim}
\end{tcolorbox}
        
    \begin{tcolorbox}[breakable, size=fbox, boxrule=1pt, pad at break*=1mm,colback=cellbackground, colframe=cellborder]
\prompt{In}{incolor}{ }{\boxspacing}
\begin{Verbatim}[commandchars=\\\{\}]
\PY{n}{segment} \PY{o}{=} \PY{n}{wave}\PY{o}{.}\PY{n}{segment}\PY{p}{(}\PY{n}{start}\PY{o}{=}\PY{l+m+mf}{1.5}\PY{p}{,} \PY{n}{duration}\PY{o}{=}\PY{l+m+mf}{1.0}\PY{p}{)}
\PY{n}{segment}\PY{o}{.}\PY{n}{make\PYZus{}audio}\PY{p}{(}\PY{p}{)}
\end{Verbatim}
\end{tcolorbox}

            \begin{tcolorbox}[breakable, size=fbox, boxrule=.5pt, pad at break*=1mm, opacityfill=0]
\prompt{Out}{outcolor}{ }{\boxspacing}
\begin{Verbatim}[commandchars=\\\{\}]
<IPython.lib.display.Audio object>
\end{Verbatim}
\end{tcolorbox}
        
    \begin{tcolorbox}[breakable, size=fbox, boxrule=1pt, pad at break*=1mm,colback=cellbackground, colframe=cellborder]
\prompt{In}{incolor}{ }{\boxspacing}
\begin{Verbatim}[commandchars=\\\{\}]
\PY{n}{spectrum} \PY{o}{=} \PY{n}{segment}\PY{o}{.}\PY{n}{make\PYZus{}spectrum}\PY{p}{(}\PY{p}{)}
\PY{n}{spectrum}\PY{o}{.}\PY{n}{plot\PYZus{}power}\PY{p}{(}\PY{p}{)}
\PY{n}{decorate}\PY{p}{(}\PY{n}{xlabel}\PY{o}{=}\PY{l+s+s1}{\PYZsq{}}\PY{l+s+s1}{Frequency (Hz)}\PY{l+s+s1}{\PYZsq{}}\PY{p}{,}
         \PY{n}{ylabel}\PY{o}{=}\PY{l+s+s1}{\PYZsq{}}\PY{l+s+s1}{Power}\PY{l+s+s1}{\PYZsq{}}\PY{p}{)}
\end{Verbatim}
\end{tcolorbox}

    \begin{center}
    \adjustimage{max size={0.9\linewidth}{0.9\paperheight}}{Vasiliev_lab4_files/Vasiliev_lab4_7_0.png}
    \end{center}
    { \hspace*{\fill} \\}
    
    Из полученного графика видно, что зависимость падения амплитуды от
частоты отдаленно напоминает розовый или белый шум (то есть линейна).
Взглянем на спектр мощности в лога- рифмическом масштабе.

    \begin{tcolorbox}[breakable, size=fbox, boxrule=1pt, pad at break*=1mm,colback=cellbackground, colframe=cellborder]
\prompt{In}{incolor}{ }{\boxspacing}
\begin{Verbatim}[commandchars=\\\{\}]
\PY{n}{spectrum}\PY{o}{.}\PY{n}{plot\PYZus{}power}\PY{p}{(}\PY{p}{)}

\PY{n}{loglog} \PY{o}{=} \PY{n+nb}{dict}\PY{p}{(}\PY{n}{xscale}\PY{o}{=}\PY{l+s+s1}{\PYZsq{}}\PY{l+s+s1}{log}\PY{l+s+s1}{\PYZsq{}}\PY{p}{,} \PY{n}{yscale}\PY{o}{=}\PY{l+s+s1}{\PYZsq{}}\PY{l+s+s1}{log}\PY{l+s+s1}{\PYZsq{}}\PY{p}{)}
\PY{n}{decorate}\PY{p}{(}\PY{n}{xlabel}\PY{o}{=}\PY{l+s+s1}{\PYZsq{}}\PY{l+s+s1}{Frequency (Hz)}\PY{l+s+s1}{\PYZsq{}}\PY{p}{,}
         \PY{n}{ylabel}\PY{o}{=}\PY{l+s+s1}{\PYZsq{}}\PY{l+s+s1}{Power}\PY{l+s+s1}{\PYZsq{}}\PY{p}{,} 
         \PY{o}{*}\PY{o}{*}\PY{n}{loglog}\PY{p}{)}
\end{Verbatim}
\end{tcolorbox}

    \begin{center}
    \adjustimage{max size={0.9\linewidth}{0.9\paperheight}}{Vasiliev_lab4_files/Vasiliev_lab4_9_0.png}
    \end{center}
    { \hspace*{\fill} \\}
    
    График сначала возрастает, а потом снова уменьшается.

    \begin{tcolorbox}[breakable, size=fbox, boxrule=1pt, pad at break*=1mm,colback=cellbackground, colframe=cellborder]
\prompt{In}{incolor}{ }{\boxspacing}
\begin{Verbatim}[commandchars=\\\{\}]
\PY{n}{segment2} \PY{o}{=} \PY{n}{wave}\PY{o}{.}\PY{n}{segment}\PY{p}{(}\PY{n}{start}\PY{o}{=}\PY{l+m+mf}{2.5}\PY{p}{,} \PY{n}{duration}\PY{o}{=}\PY{l+m+mf}{1.0}\PY{p}{)}
\PY{n}{segment2}\PY{o}{.}\PY{n}{make\PYZus{}audio}\PY{p}{(}\PY{p}{)}
\end{Verbatim}
\end{tcolorbox}

            \begin{tcolorbox}[breakable, size=fbox, boxrule=.5pt, pad at break*=1mm, opacityfill=0]
\prompt{Out}{outcolor}{ }{\boxspacing}
\begin{Verbatim}[commandchars=\\\{\}]
<IPython.lib.display.Audio object>
\end{Verbatim}
\end{tcolorbox}
        
    \begin{tcolorbox}[breakable, size=fbox, boxrule=1pt, pad at break*=1mm,colback=cellbackground, colframe=cellborder]
\prompt{In}{incolor}{ }{\boxspacing}
\begin{Verbatim}[commandchars=\\\{\}]
\PY{n}{spectrum2} \PY{o}{=} \PY{n}{segment2}\PY{o}{.}\PY{n}{make\PYZus{}spectrum}\PY{p}{(}\PY{p}{)}

\PY{n}{spectrum}\PY{o}{.}\PY{n}{plot\PYZus{}power}\PY{p}{(}\PY{n}{alpha}\PY{o}{=}\PY{l+m+mf}{0.5}\PY{p}{)}
\PY{n}{spectrum2}\PY{o}{.}\PY{n}{plot\PYZus{}power}\PY{p}{(}\PY{n}{alpha}\PY{o}{=}\PY{l+m+mf}{0.5}\PY{p}{)}
\PY{n}{decorate}\PY{p}{(}\PY{n}{xlabel}\PY{o}{=}\PY{l+s+s1}{\PYZsq{}}\PY{l+s+s1}{Frequency (Hz)}\PY{l+s+s1}{\PYZsq{}}\PY{p}{,}
         \PY{n}{ylabel}\PY{o}{=}\PY{l+s+s1}{\PYZsq{}}\PY{l+s+s1}{Power}\PY{l+s+s1}{\PYZsq{}}\PY{p}{)}
\end{Verbatim}
\end{tcolorbox}

    \begin{center}
    \adjustimage{max size={0.9\linewidth}{0.9\paperheight}}{Vasiliev_lab4_files/Vasiliev_lab4_12_0.png}
    \end{center}
    { \hspace*{\fill} \\}
    
    \begin{tcolorbox}[breakable, size=fbox, boxrule=1pt, pad at break*=1mm,colback=cellbackground, colframe=cellborder]
\prompt{In}{incolor}{ }{\boxspacing}
\begin{Verbatim}[commandchars=\\\{\}]
\PY{n}{spectrum}\PY{o}{.}\PY{n}{plot\PYZus{}power}\PY{p}{(}\PY{n}{alpha}\PY{o}{=}\PY{l+m+mf}{0.5}\PY{p}{)}
\PY{n}{spectrum2}\PY{o}{.}\PY{n}{plot\PYZus{}power}\PY{p}{(}\PY{n}{alpha}\PY{o}{=}\PY{l+m+mf}{0.5}\PY{p}{)}
\PY{n}{decorate}\PY{p}{(}\PY{n}{xlabel}\PY{o}{=}\PY{l+s+s1}{\PYZsq{}}\PY{l+s+s1}{Frequency (Hz)}\PY{l+s+s1}{\PYZsq{}}\PY{p}{,}
         \PY{n}{ylabel}\PY{o}{=}\PY{l+s+s1}{\PYZsq{}}\PY{l+s+s1}{Power}\PY{l+s+s1}{\PYZsq{}}\PY{p}{,}
         \PY{o}{*}\PY{o}{*}\PY{n}{loglog}\PY{p}{)}
\end{Verbatim}
\end{tcolorbox}

    \begin{center}
    \adjustimage{max size={0.9\linewidth}{0.9\paperheight}}{Vasiliev_lab4_files/Vasiliev_lab4_13_0.png}
    \end{center}
    { \hspace*{\fill} \\}
    
    \begin{tcolorbox}[breakable, size=fbox, boxrule=1pt, pad at break*=1mm,colback=cellbackground, colframe=cellborder]
\prompt{In}{incolor}{ }{\boxspacing}
\begin{Verbatim}[commandchars=\\\{\}]
\PY{n}{segment}\PY{o}{.}\PY{n}{make\PYZus{}spectrogram}\PY{p}{(}\PY{l+m+mi}{512}\PY{p}{)}\PY{o}{.}\PY{n}{plot}\PY{p}{(}\PY{n}{high}\PY{o}{=}\PY{l+m+mi}{5000}\PY{p}{)}
\PY{n}{decorate}\PY{p}{(}\PY{n}{xlabel}\PY{o}{=}\PY{l+s+s1}{\PYZsq{}}\PY{l+s+s1}{Time(s)}\PY{l+s+s1}{\PYZsq{}}\PY{p}{,} \PY{n}{ylabel}\PY{o}{=}\PY{l+s+s1}{\PYZsq{}}\PY{l+s+s1}{Frequency (Hz)}\PY{l+s+s1}{\PYZsq{}}\PY{p}{)}
\end{Verbatim}
\end{tcolorbox}

    \begin{center}
    \adjustimage{max size={0.9\linewidth}{0.9\paperheight}}{Vasiliev_lab4_files/Vasiliev_lab4_14_0.png}
    \end{center}
    { \hspace*{\fill} \\}
    
    Создадим метод barlet\_method, который будет брать сигнал, разделять его
на сегменты и вычислять спектр мощности для каждого сегмента и находить
среднее по сегментам. Для этого нужно в аргументы функции передадим сам
сигнал и желаемую длину каждого сегмента. Затем вычислим спектр sp и
выделим из него отдельные спекрты specs. Затем выделим массив psds
мощностей из каждого полученного спектра. Вычислим среднюю мощность hs
начального сигнала.

    \begin{tcolorbox}[breakable, size=fbox, boxrule=1pt, pad at break*=1mm,colback=cellbackground, colframe=cellborder]
\prompt{In}{incolor}{ }{\boxspacing}
\begin{Verbatim}[commandchars=\\\{\}]
\PY{k+kn}{from} \PY{n+nn}{thinkdsp} \PY{k+kn}{import} \PY{n}{Spectrum}

\PY{k}{def} \PY{n+nf}{bartlett\PYZus{}method}\PY{p}{(}\PY{n}{wave}\PY{p}{,} \PY{n}{seg\PYZus{}length}\PY{o}{=}\PY{l+m+mi}{512}\PY{p}{,} \PY{n}{win\PYZus{}flag}\PY{o}{=}\PY{k+kc}{True}\PY{p}{)}\PY{p}{:}
    
    \PY{n}{spectro} \PY{o}{=} \PY{n}{wave}\PY{o}{.}\PY{n}{make\PYZus{}spectrogram}\PY{p}{(}\PY{n}{seg\PYZus{}length}\PY{p}{,} \PY{n}{win\PYZus{}flag}\PY{p}{)}
    \PY{n}{spectrums} \PY{o}{=} \PY{n}{spectro}\PY{o}{.}\PY{n}{spec\PYZus{}map}\PY{o}{.}\PY{n}{values}\PY{p}{(}\PY{p}{)}

    \PY{n}{psds} \PY{o}{=} \PY{p}{[}\PY{n}{spectrum}\PY{o}{.}\PY{n}{power} \PY{k}{for} \PY{n}{spectrum} \PY{o+ow}{in} \PY{n}{spectrums}\PY{p}{]}

    \PY{n}{hs} \PY{o}{=} \PY{n}{np}\PY{o}{.}\PY{n}{sqrt}\PY{p}{(}\PY{n+nb}{sum}\PY{p}{(}\PY{n}{psds}\PY{p}{)} \PY{o}{/} \PY{n+nb}{len}\PY{p}{(}\PY{n}{psds}\PY{p}{)}\PY{p}{)}
    \PY{n}{fs} \PY{o}{=} \PY{n+nb}{next}\PY{p}{(}\PY{n+nb}{iter}\PY{p}{(}\PY{n}{spectrums}\PY{p}{)}\PY{p}{)}\PY{o}{.}\PY{n}{fs}

    \PY{n}{spectrum} \PY{o}{=} \PY{n}{Spectrum}\PY{p}{(}\PY{n}{hs}\PY{p}{,} \PY{n}{fs}\PY{p}{,} \PY{n}{wave}\PY{o}{.}\PY{n}{framerate}\PY{p}{)}
    \PY{k}{return} \PY{n}{spectrum}
\end{Verbatim}
\end{tcolorbox}

    \begin{tcolorbox}[breakable, size=fbox, boxrule=1pt, pad at break*=1mm,colback=cellbackground, colframe=cellborder]
\prompt{In}{incolor}{ }{\boxspacing}
\begin{Verbatim}[commandchars=\\\{\}]
\PY{n}{psd} \PY{o}{=} \PY{n}{bartlett\PYZus{}method}\PY{p}{(}\PY{n}{segment}\PY{p}{)}
\PY{n}{psd2} \PY{o}{=} \PY{n}{bartlett\PYZus{}method}\PY{p}{(}\PY{n}{segment2}\PY{p}{)}

\PY{n}{psd}\PY{o}{.}\PY{n}{plot\PYZus{}power}\PY{p}{(}\PY{p}{)}
\PY{n}{psd2}\PY{o}{.}\PY{n}{plot\PYZus{}power}\PY{p}{(}\PY{p}{)}

\PY{n}{decorate}\PY{p}{(}\PY{n}{xlabel}\PY{o}{=}\PY{l+s+s1}{\PYZsq{}}\PY{l+s+s1}{Frequency (Hz)}\PY{l+s+s1}{\PYZsq{}}\PY{p}{,} 
         \PY{n}{ylabel}\PY{o}{=}\PY{l+s+s1}{\PYZsq{}}\PY{l+s+s1}{Power}\PY{l+s+s1}{\PYZsq{}}\PY{p}{,} 
         \PY{o}{*}\PY{o}{*}\PY{n}{loglog}\PY{p}{)}
\end{Verbatim}
\end{tcolorbox}

    \begin{center}
    \adjustimage{max size={0.9\linewidth}{0.9\paperheight}}{Vasiliev_lab4_files/Vasiliev_lab4_17_0.png}
    \end{center}
    { \hspace*{\fill} \\}
    
    \hypertarget{ux443ux43fux440ux430ux436ux43dux435ux43dux438ux435-3}{%
\subsection{Упражнение
3}\label{ux443ux43fux440ux430ux436ux43dux435ux43dux438ux435-3}}

    BTC.csv файл содержит исторические данные о ежедневной цене биткоина за
последние пол- года. Откроем этот файл и вычислим спектр цен как функцию
от времени.

    \begin{tcolorbox}[breakable, size=fbox, boxrule=1pt, pad at break*=1mm,colback=cellbackground, colframe=cellborder]
\prompt{In}{incolor}{ }{\boxspacing}
\begin{Verbatim}[commandchars=\\\{\}]
\PY{k}{if} \PY{o+ow}{not} \PY{n}{os}\PY{o}{.}\PY{n}{path}\PY{o}{.}\PY{n}{exists}\PY{p}{(}\PY{l+s+s1}{\PYZsq{}}\PY{l+s+s1}{BTC\PYZus{}USD\PYZus{}2013\PYZhy{}10\PYZhy{}01\PYZus{}2020\PYZhy{}03\PYZhy{}26\PYZhy{}CoinDesk.csv}\PY{l+s+s1}{\PYZsq{}}\PY{p}{)}\PY{p}{:}
    \PY{err}{!}\PY{n}{wget} \PY{n}{https}\PY{p}{:}\PY{o}{/}\PY{o}{/}\PY{n}{github}\PY{o}{.}\PY{n}{com}\PY{o}{/}\PY{n}{AllenDowney}\PY{o}{/}\PY{n}{ThinkDSP}\PY{o}{/}\PY{n}{raw}\PY{o}{/}\PY{n}{master}\PY{o}{/}\PY{n}{code}\PY{o}{/}\PY{n}{BTC\PYZus{}USD\PYZus{}2013}\PY{o}{\PYZhy{}}\PY{l+m+mi}{10}\PY{o}{\PYZhy{}}\PY{l+m+mi}{01\PYZus{}2020}\PY{o}{\PYZhy{}}\PY{l+m+mi}{03}\PY{o}{\PYZhy{}}\PY{l+m+mi}{26}\PY{o}{\PYZhy{}}\PY{n}{CoinDesk}\PY{o}{.}\PY{n}{csv}
\end{Verbatim}
\end{tcolorbox}

    \begin{Verbatim}[commandchars=\\\{\}]
--2022-04-08 12:15:08--  https://github.com/AllenDowney/ThinkDSP/raw/master/code
/BTC\_USD\_2013-10-01\_2020-03-26-CoinDesk.csv
Resolving github.com (github.com){\ldots} 140.82.112.3
Connecting to github.com (github.com)|140.82.112.3|:443{\ldots} connected.
HTTP request sent, awaiting response{\ldots} 302 Found
Location: https://raw.githubusercontent.com/AllenDowney/ThinkDSP/master/code/BTC
\_USD\_2013-10-01\_2020-03-26-CoinDesk.csv [following]
--2022-04-08 12:15:09--  https://raw.githubusercontent.com/AllenDowney/ThinkDSP/
master/code/BTC\_USD\_2013-10-01\_2020-03-26-CoinDesk.csv
Resolving raw.githubusercontent.com (raw.githubusercontent.com){\ldots}
185.199.111.133, 185.199.108.133, 185.199.110.133, {\ldots}
Connecting to raw.githubusercontent.com
(raw.githubusercontent.com)|185.199.111.133|:443{\ldots} connected.
HTTP request sent, awaiting response{\ldots} 200 OK
Length: 143622 (140K) [text/plain]
Saving to: ‘BTC\_USD\_2013-10-01\_2020-03-26-CoinDesk.csv’

BTC\_USD\_2013-10-01\_ 100\%[===================>] 140.26K  --.-KB/s    in 0.02s

2022-04-08 12:15:09 (6.52 MB/s) - ‘BTC\_USD\_2013-10-01\_2020-03-26-CoinDesk.csv’
saved [143622/143622]

    \end{Verbatim}

    \begin{tcolorbox}[breakable, size=fbox, boxrule=1pt, pad at break*=1mm,colback=cellbackground, colframe=cellborder]
\prompt{In}{incolor}{ }{\boxspacing}
\begin{Verbatim}[commandchars=\\\{\}]
\PY{k+kn}{import} \PY{n+nn}{pandas} \PY{k}{as} \PY{n+nn}{pd}

\PY{n}{df} \PY{o}{=} \PY{n}{pd}\PY{o}{.}\PY{n}{read\PYZus{}csv}\PY{p}{(}\PY{l+s+s1}{\PYZsq{}}\PY{l+s+s1}{BTC\PYZus{}USD\PYZus{}2013\PYZhy{}10\PYZhy{}01\PYZus{}2020\PYZhy{}03\PYZhy{}26\PYZhy{}CoinDesk.csv}\PY{l+s+s1}{\PYZsq{}}\PY{p}{,} 
                 \PY{n}{parse\PYZus{}dates}\PY{o}{=}\PY{p}{[}\PY{l+m+mi}{0}\PY{p}{]}\PY{p}{)}
\PY{n}{df}
\end{Verbatim}
\end{tcolorbox}

            \begin{tcolorbox}[breakable, size=fbox, boxrule=.5pt, pad at break*=1mm, opacityfill=0]
\prompt{Out}{outcolor}{ }{\boxspacing}
\begin{Verbatim}[commandchars=\\\{\}]
     Currency        Date  Closing Price (USD)  24h Open (USD)  \textbackslash{}
0         BTC  2013-10-01           123.654990      124.304660
1         BTC  2013-10-02           125.455000      123.654990
2         BTC  2013-10-03           108.584830      125.455000
3         BTC  2013-10-04           118.674660      108.584830
4         BTC  2013-10-05           121.338660      118.674660
{\ldots}       {\ldots}         {\ldots}                  {\ldots}             {\ldots}
2354      BTC  2020-03-22          5884.340133     6187.042146
2355      BTC  2020-03-23          6455.454688     5829.352511
2356      BTC  2020-03-24          6784.318011     6455.450650
2357      BTC  2020-03-25          6706.985089     6784.325204
2358      BTC  2020-03-26          6721.495392     6697.948320

      24h High (USD)  24h Low (USD)
0         124.751660     122.563490
1         125.758500     123.633830
2         125.665660      83.328330
3         118.675000     107.058160
4         121.936330     118.005660
{\ldots}              {\ldots}            {\ldots}
2354     6431.873162    5802.553402
2355     6620.858253    5694.198299
2356     6863.602196    6406.037439
2357     6981.720386    6488.111885
2358     6796.053701    6537.856462

[2359 rows x 6 columns]
\end{Verbatim}
\end{tcolorbox}
        
    \begin{tcolorbox}[breakable, size=fbox, boxrule=1pt, pad at break*=1mm,colback=cellbackground, colframe=cellborder]
\prompt{In}{incolor}{ }{\boxspacing}
\begin{Verbatim}[commandchars=\\\{\}]
\PY{n}{ys} \PY{o}{=} \PY{n}{df}\PY{p}{[}\PY{l+s+s1}{\PYZsq{}}\PY{l+s+s1}{Closing Price (USD)}\PY{l+s+s1}{\PYZsq{}}\PY{p}{]}
\PY{n}{ts} \PY{o}{=} \PY{n}{df}\PY{o}{.}\PY{n}{index}
\end{Verbatim}
\end{tcolorbox}

    \begin{tcolorbox}[breakable, size=fbox, boxrule=1pt, pad at break*=1mm,colback=cellbackground, colframe=cellborder]
\prompt{In}{incolor}{ }{\boxspacing}
\begin{Verbatim}[commandchars=\\\{\}]
\PY{k+kn}{from} \PY{n+nn}{thinkdsp} \PY{k+kn}{import} \PY{n}{Wave}

\PY{n}{wave} \PY{o}{=} \PY{n}{Wave}\PY{p}{(}\PY{n}{ys}\PY{p}{,} \PY{n}{ts}\PY{p}{,} \PY{n}{framerate}\PY{o}{=}\PY{l+m+mi}{1}\PY{p}{)}
\PY{n}{wave}\PY{o}{.}\PY{n}{plot}\PY{p}{(}\PY{p}{)}
\PY{n}{decorate}\PY{p}{(}\PY{n}{xlabel}\PY{o}{=}\PY{l+s+s1}{\PYZsq{}}\PY{l+s+s1}{Time (days)}\PY{l+s+s1}{\PYZsq{}}\PY{p}{)}
\end{Verbatim}
\end{tcolorbox}

    \begin{center}
    \adjustimage{max size={0.9\linewidth}{0.9\paperheight}}{Vasiliev_lab4_files/Vasiliev_lab4_23_0.png}
    \end{center}
    { \hspace*{\fill} \\}
    
    \begin{tcolorbox}[breakable, size=fbox, boxrule=1pt, pad at break*=1mm,colback=cellbackground, colframe=cellborder]
\prompt{In}{incolor}{ }{\boxspacing}
\begin{Verbatim}[commandchars=\\\{\}]
\PY{n}{spectrum} \PY{o}{=} \PY{n}{wave}\PY{o}{.}\PY{n}{make\PYZus{}spectrum}\PY{p}{(}\PY{p}{)}
\PY{n}{spectrum}\PY{o}{.}\PY{n}{plot\PYZus{}power}\PY{p}{(}\PY{p}{)}
\PY{n}{decorate}\PY{p}{(}\PY{n}{xlabel}\PY{o}{=}\PY{l+s+s1}{\PYZsq{}}\PY{l+s+s1}{Frequency (1/days)}\PY{l+s+s1}{\PYZsq{}}\PY{p}{,}
         \PY{n}{ylabel}\PY{o}{=}\PY{l+s+s1}{\PYZsq{}}\PY{l+s+s1}{Power}\PY{l+s+s1}{\PYZsq{}}\PY{p}{,} 
         \PY{o}{*}\PY{o}{*}\PY{n}{loglog}\PY{p}{)}
\end{Verbatim}
\end{tcolorbox}

    \begin{center}
    \adjustimage{max size={0.9\linewidth}{0.9\paperheight}}{Vasiliev_lab4_files/Vasiliev_lab4_24_0.png}
    \end{center}
    { \hspace*{\fill} \\}
    
    \begin{tcolorbox}[breakable, size=fbox, boxrule=1pt, pad at break*=1mm,colback=cellbackground, colframe=cellborder]
\prompt{In}{incolor}{ }{\boxspacing}
\begin{Verbatim}[commandchars=\\\{\}]
\PY{n}{spectrum}\PY{o}{.}\PY{n}{estimate\PYZus{}slope}\PY{p}{(}\PY{p}{)}\PY{p}{[}\PY{l+m+mi}{0}\PY{p}{]}
\end{Verbatim}
\end{tcolorbox}

            \begin{tcolorbox}[breakable, size=fbox, boxrule=.5pt, pad at break*=1mm, opacityfill=0]
\prompt{Out}{outcolor}{ }{\boxspacing}
\begin{Verbatim}[commandchars=\\\{\}]
-1.7332540936758951
\end{Verbatim}
\end{tcolorbox}
        
    slope примерно равен -2, что соответствует наклону красного шума.

    \hypertarget{ux443ux43fux440ux430ux436ux43dux435ux43dux438ux435-4}{%
\subsection{Упражнение
4}\label{ux443ux43fux440ux430ux436ux43dux435ux43dux438ux435-4}}

    Напишем класс UncorrelatedPoissonNoise, который наследуется от класса
thinkdsp.\_Noise, который моделирует некоррелированный пуассонвский шум
(UP). Для этого переопределим функцию evaluate, в которой используем
метод np.random.poisson(). Параметр этой функ- ции lam - среднее число
частиц за время каждого интервала.

    \begin{tcolorbox}[breakable, size=fbox, boxrule=1pt, pad at break*=1mm,colback=cellbackground, colframe=cellborder]
\prompt{In}{incolor}{ }{\boxspacing}
\begin{Verbatim}[commandchars=\\\{\}]
\PY{k+kn}{from} \PY{n+nn}{thinkdsp} \PY{k+kn}{import} \PY{n}{Noise}

\PY{k}{class} \PY{n+nc}{UncorrelatedPoissonNoise}\PY{p}{(}\PY{n}{Noise}\PY{p}{)}\PY{p}{:}

    \PY{k}{def} \PY{n+nf}{evaluate}\PY{p}{(}\PY{n+nb+bp}{self}\PY{p}{,} \PY{n}{ts}\PY{p}{)}\PY{p}{:}

        \PY{n}{ys} \PY{o}{=} \PY{n}{np}\PY{o}{.}\PY{n}{random}\PY{o}{.}\PY{n}{poisson}\PY{p}{(}\PY{n+nb+bp}{self}\PY{o}{.}\PY{n}{amp}\PY{p}{,} \PY{n+nb}{len}\PY{p}{(}\PY{n}{ts}\PY{p}{)}\PY{p}{)}
        \PY{k}{return} \PY{n}{ys}
\end{Verbatim}
\end{tcolorbox}

    Сгенерируем сигнал с маленькой амплитудой (0.001) на основе этого
класса. Ожидается услы- шать звук, как у счетчика Гейгера.

    \begin{tcolorbox}[breakable, size=fbox, boxrule=1pt, pad at break*=1mm,colback=cellbackground, colframe=cellborder]
\prompt{In}{incolor}{ }{\boxspacing}
\begin{Verbatim}[commandchars=\\\{\}]
\PY{n}{amp} \PY{o}{=} \PY{l+m+mf}{0.001}
\PY{n}{framerate} \PY{o}{=} \PY{l+m+mi}{10000}
\PY{n}{duration} \PY{o}{=} \PY{l+m+mi}{1}

\PY{n}{signal} \PY{o}{=} \PY{n}{UncorrelatedPoissonNoise}\PY{p}{(}\PY{n}{amp}\PY{o}{=}\PY{n}{amp}\PY{p}{)}
\PY{n}{wave} \PY{o}{=} \PY{n}{signal}\PY{o}{.}\PY{n}{make\PYZus{}wave}\PY{p}{(}\PY{n}{duration}\PY{o}{=}\PY{n}{duration}\PY{p}{,} \PY{n}{framerate}\PY{o}{=}\PY{n}{framerate}\PY{p}{)}
\PY{n}{wave}\PY{o}{.}\PY{n}{make\PYZus{}audio}\PY{p}{(}\PY{p}{)}
\end{Verbatim}
\end{tcolorbox}

            \begin{tcolorbox}[breakable, size=fbox, boxrule=.5pt, pad at break*=1mm, opacityfill=0]
\prompt{Out}{outcolor}{ }{\boxspacing}
\begin{Verbatim}[commandchars=\\\{\}]
<IPython.lib.display.Audio object>
\end{Verbatim}
\end{tcolorbox}
        
    \begin{tcolorbox}[breakable, size=fbox, boxrule=1pt, pad at break*=1mm,colback=cellbackground, colframe=cellborder]
\prompt{In}{incolor}{ }{\boxspacing}
\begin{Verbatim}[commandchars=\\\{\}]
\PY{n}{expected} \PY{o}{=} \PY{n}{amp} \PY{o}{*} \PY{n}{framerate} \PY{o}{*} \PY{n}{duration}
\PY{n}{actual} \PY{o}{=} \PY{n+nb}{sum}\PY{p}{(}\PY{n}{wave}\PY{o}{.}\PY{n}{ys}\PY{p}{)}
\PY{n+nb}{print}\PY{p}{(}\PY{n}{expected}\PY{p}{,} \PY{n}{actual}\PY{p}{)}
\end{Verbatim}
\end{tcolorbox}

    \begin{Verbatim}[commandchars=\\\{\}]
10.0 8
    \end{Verbatim}

    \begin{tcolorbox}[breakable, size=fbox, boxrule=1pt, pad at break*=1mm,colback=cellbackground, colframe=cellborder]
\prompt{In}{incolor}{ }{\boxspacing}
\begin{Verbatim}[commandchars=\\\{\}]
\PY{n}{wave}\PY{o}{.}\PY{n}{plot}\PY{p}{(}\PY{p}{)}
\end{Verbatim}
\end{tcolorbox}

    \begin{center}
    \adjustimage{max size={0.9\linewidth}{0.9\paperheight}}{Vasiliev_lab4_files/Vasiliev_lab4_33_0.png}
    \end{center}
    { \hspace*{\fill} \\}
    
    \begin{tcolorbox}[breakable, size=fbox, boxrule=1pt, pad at break*=1mm,colback=cellbackground, colframe=cellborder]
\prompt{In}{incolor}{ }{\boxspacing}
\begin{Verbatim}[commandchars=\\\{\}]
\PY{n}{spectrum} \PY{o}{=} \PY{n}{wave}\PY{o}{.}\PY{n}{make\PYZus{}spectrum}\PY{p}{(}\PY{p}{)}
\PY{n}{spectrum}\PY{o}{.}\PY{n}{plot\PYZus{}power}\PY{p}{(}\PY{p}{)}
\PY{n}{decorate}\PY{p}{(}\PY{n}{xlabel}\PY{o}{=}\PY{l+s+s1}{\PYZsq{}}\PY{l+s+s1}{Frequency (Hz)}\PY{l+s+s1}{\PYZsq{}}\PY{p}{,}
         \PY{n}{ylabel}\PY{o}{=}\PY{l+s+s1}{\PYZsq{}}\PY{l+s+s1}{Power}\PY{l+s+s1}{\PYZsq{}}\PY{p}{,}
         \PY{o}{*}\PY{o}{*}\PY{n}{loglog}\PY{p}{)}
\end{Verbatim}
\end{tcolorbox}

    \begin{center}
    \adjustimage{max size={0.9\linewidth}{0.9\paperheight}}{Vasiliev_lab4_files/Vasiliev_lab4_34_0.png}
    \end{center}
    { \hspace*{\fill} \\}
    
    \begin{tcolorbox}[breakable, size=fbox, boxrule=1pt, pad at break*=1mm,colback=cellbackground, colframe=cellborder]
\prompt{In}{incolor}{ }{\boxspacing}
\begin{Verbatim}[commandchars=\\\{\}]
\PY{n}{spectrum}\PY{o}{.}\PY{n}{estimate\PYZus{}slope}\PY{p}{(}\PY{p}{)}\PY{o}{.}\PY{n}{slope}
\end{Verbatim}
\end{tcolorbox}

    \begin{Verbatim}[commandchars=\\\{\}]
/content/thinkdsp.py:294: RuntimeWarning: divide by zero encountered in log
  y = np.log(self.power[1:])
/usr/local/lib/python3.7/dist-packages/numpy/lib/function\_base.py:2536:
RuntimeWarning: invalid value encountered in subtract
  X -= avg[:, None]
    \end{Verbatim}

            \begin{tcolorbox}[breakable, size=fbox, boxrule=.5pt, pad at break*=1mm, opacityfill=0]
\prompt{Out}{outcolor}{ }{\boxspacing}
\begin{Verbatim}[commandchars=\\\{\}]
nan
\end{Verbatim}
\end{tcolorbox}
        
    Создадим такой же сигнал, но с большей амплитудой.

    \begin{tcolorbox}[breakable, size=fbox, boxrule=1pt, pad at break*=1mm,colback=cellbackground, colframe=cellborder]
\prompt{In}{incolor}{ }{\boxspacing}
\begin{Verbatim}[commandchars=\\\{\}]
\PY{n}{amp} \PY{o}{=} \PY{l+m+mi}{1}
\PY{n}{framerate} \PY{o}{=} \PY{l+m+mi}{10000}
\PY{n}{duration} \PY{o}{=} \PY{l+m+mi}{1}

\PY{n}{signal} \PY{o}{=} \PY{n}{UncorrelatedPoissonNoise}\PY{p}{(}\PY{n}{amp}\PY{o}{=}\PY{n}{amp}\PY{p}{)}
\PY{n}{wave} \PY{o}{=} \PY{n}{signal}\PY{o}{.}\PY{n}{make\PYZus{}wave}\PY{p}{(}\PY{n}{duration}\PY{o}{=}\PY{n}{duration}\PY{p}{,} \PY{n}{framerate}\PY{o}{=}\PY{n}{framerate}\PY{p}{)}
\PY{n}{wave}\PY{o}{.}\PY{n}{make\PYZus{}audio}\PY{p}{(}\PY{p}{)}
\end{Verbatim}
\end{tcolorbox}

            \begin{tcolorbox}[breakable, size=fbox, boxrule=.5pt, pad at break*=1mm, opacityfill=0]
\prompt{Out}{outcolor}{ }{\boxspacing}
\begin{Verbatim}[commandchars=\\\{\}]
<IPython.lib.display.Audio object>
\end{Verbatim}
\end{tcolorbox}
        
    \begin{tcolorbox}[breakable, size=fbox, boxrule=1pt, pad at break*=1mm,colback=cellbackground, colframe=cellborder]
\prompt{In}{incolor}{ }{\boxspacing}
\begin{Verbatim}[commandchars=\\\{\}]
\PY{n}{wave}\PY{o}{.}\PY{n}{plot}\PY{p}{(}\PY{p}{)}
\end{Verbatim}
\end{tcolorbox}

    \begin{center}
    \adjustimage{max size={0.9\linewidth}{0.9\paperheight}}{Vasiliev_lab4_files/Vasiliev_lab4_38_0.png}
    \end{center}
    { \hspace*{\fill} \\}
    
    \begin{tcolorbox}[breakable, size=fbox, boxrule=1pt, pad at break*=1mm,colback=cellbackground, colframe=cellborder]
\prompt{In}{incolor}{ }{\boxspacing}
\begin{Verbatim}[commandchars=\\\{\}]
\PY{k+kn}{import} \PY{n+nn}{matplotlib}\PY{n+nn}{.}\PY{n+nn}{pyplot} \PY{k}{as} \PY{n+nn}{plt}

\PY{k}{def} \PY{n+nf}{normal\PYZus{}prob\PYZus{}plot}\PY{p}{(}\PY{n}{sample}\PY{p}{,} \PY{n}{fit\PYZus{}color}\PY{o}{=}\PY{l+s+s1}{\PYZsq{}}\PY{l+s+s1}{0.8}\PY{l+s+s1}{\PYZsq{}}\PY{p}{,} \PY{o}{*}\PY{o}{*}\PY{n}{options}\PY{p}{)}\PY{p}{:}
    \PY{n}{n} \PY{o}{=} \PY{n+nb}{len}\PY{p}{(}\PY{n}{sample}\PY{p}{)}
    \PY{n}{xs} \PY{o}{=} \PY{n}{np}\PY{o}{.}\PY{n}{random}\PY{o}{.}\PY{n}{normal}\PY{p}{(}\PY{l+m+mi}{0}\PY{p}{,} \PY{l+m+mi}{1}\PY{p}{,} \PY{n}{n}\PY{p}{)}
    \PY{n}{xs}\PY{o}{.}\PY{n}{sort}\PY{p}{(}\PY{p}{)}
    
    \PY{n}{ys} \PY{o}{=} \PY{n}{np}\PY{o}{.}\PY{n}{sort}\PY{p}{(}\PY{n}{sample}\PY{p}{)}
    
    \PY{n}{mean}\PY{p}{,} \PY{n}{std} \PY{o}{=} \PY{n}{np}\PY{o}{.}\PY{n}{mean}\PY{p}{(}\PY{n}{sample}\PY{p}{)}\PY{p}{,} \PY{n}{np}\PY{o}{.}\PY{n}{std}\PY{p}{(}\PY{n}{sample}\PY{p}{)}
    \PY{n}{fit\PYZus{}ys} \PY{o}{=} \PY{n}{mean} \PY{o}{+} \PY{n}{std} \PY{o}{*} \PY{n}{xs}
    \PY{n}{plt}\PY{o}{.}\PY{n}{plot}\PY{p}{(}\PY{n}{xs}\PY{p}{,} \PY{n}{fit\PYZus{}ys}\PY{p}{,} \PY{n}{color}\PY{o}{=}\PY{l+s+s1}{\PYZsq{}}\PY{l+s+s1}{gray}\PY{l+s+s1}{\PYZsq{}}\PY{p}{,} \PY{n}{alpha}\PY{o}{=}\PY{l+m+mf}{0.5}\PY{p}{,} \PY{n}{label}\PY{o}{=}\PY{l+s+s1}{\PYZsq{}}\PY{l+s+s1}{model}\PY{l+s+s1}{\PYZsq{}}\PY{p}{)}

    \PY{n}{plt}\PY{o}{.}\PY{n}{plot}\PY{p}{(}\PY{n}{xs}\PY{p}{,} \PY{n}{ys}\PY{p}{,} \PY{o}{*}\PY{o}{*}\PY{n}{options}\PY{p}{)}
\end{Verbatim}
\end{tcolorbox}

    \begin{tcolorbox}[breakable, size=fbox, boxrule=1pt, pad at break*=1mm,colback=cellbackground, colframe=cellborder]
\prompt{In}{incolor}{ }{\boxspacing}
\begin{Verbatim}[commandchars=\\\{\}]
\PY{n}{spectrum} \PY{o}{=} \PY{n}{wave}\PY{o}{.}\PY{n}{make\PYZus{}spectrum}\PY{p}{(}\PY{p}{)}
\PY{n}{spectrum}\PY{o}{.}\PY{n}{hs}\PY{p}{[}\PY{l+m+mi}{0}\PY{p}{]} \PY{o}{=} \PY{l+m+mi}{0}

\PY{n}{normal\PYZus{}prob\PYZus{}plot}\PY{p}{(}\PY{n}{spectrum}\PY{o}{.}\PY{n}{real}\PY{p}{,} \PY{n}{label}\PY{o}{=}\PY{l+s+s1}{\PYZsq{}}\PY{l+s+s1}{real}\PY{l+s+s1}{\PYZsq{}}\PY{p}{)}
\PY{n}{decorate}\PY{p}{(}\PY{n}{xlabel}\PY{o}{=}\PY{l+s+s1}{\PYZsq{}}\PY{l+s+s1}{Normal sample}\PY{l+s+s1}{\PYZsq{}}\PY{p}{,}
        \PY{n}{ylabel}\PY{o}{=}\PY{l+s+s1}{\PYZsq{}}\PY{l+s+s1}{Power}\PY{l+s+s1}{\PYZsq{}}\PY{p}{)}
\end{Verbatim}
\end{tcolorbox}

    \begin{center}
    \adjustimage{max size={0.9\linewidth}{0.9\paperheight}}{Vasiliev_lab4_files/Vasiliev_lab4_40_0.png}
    \end{center}
    { \hspace*{\fill} \\}
    
    \begin{tcolorbox}[breakable, size=fbox, boxrule=1pt, pad at break*=1mm,colback=cellbackground, colframe=cellborder]
\prompt{In}{incolor}{ }{\boxspacing}
\begin{Verbatim}[commandchars=\\\{\}]
\PY{n}{normal\PYZus{}prob\PYZus{}plot}\PY{p}{(}\PY{n}{spectrum}\PY{o}{.}\PY{n}{imag}\PY{p}{,} \PY{n}{label}\PY{o}{=}\PY{l+s+s1}{\PYZsq{}}\PY{l+s+s1}{imag}\PY{l+s+s1}{\PYZsq{}}\PY{p}{,} \PY{n}{color}\PY{o}{=}\PY{l+s+s1}{\PYZsq{}}\PY{l+s+s1}{C1}\PY{l+s+s1}{\PYZsq{}}\PY{p}{)}
\PY{n}{decorate}\PY{p}{(}\PY{n}{xlabel}\PY{o}{=}\PY{l+s+s1}{\PYZsq{}}\PY{l+s+s1}{Normal sample}\PY{l+s+s1}{\PYZsq{}}\PY{p}{)}
\end{Verbatim}
\end{tcolorbox}

    \begin{center}
    \adjustimage{max size={0.9\linewidth}{0.9\paperheight}}{Vasiliev_lab4_files/Vasiliev_lab4_41_0.png}
    \end{center}
    { \hspace*{\fill} \\}
    
    \hypertarget{ux443ux43fux440ux430ux436ux43dux435ux43dux438ux435-5}{%
\subsection{Упражнение
5}\label{ux443ux43fux440ux430ux436ux43dux435ux43dux438ux435-5}}

    \begin{tcolorbox}[breakable, size=fbox, boxrule=1pt, pad at break*=1mm,colback=cellbackground, colframe=cellborder]
\prompt{In}{incolor}{ }{\boxspacing}
\begin{Verbatim}[commandchars=\\\{\}]
\PY{n}{nrows} \PY{o}{=} \PY{l+m+mi}{100}
\PY{n}{ncols} \PY{o}{=} \PY{l+m+mi}{5}

\PY{n}{array} \PY{o}{=} \PY{n}{np}\PY{o}{.}\PY{n}{empty}\PY{p}{(}\PY{p}{(}\PY{n}{nrows}\PY{p}{,} \PY{n}{ncols}\PY{p}{)}\PY{p}{)}
\PY{n}{array}\PY{o}{.}\PY{n}{fill}\PY{p}{(}\PY{n}{np}\PY{o}{.}\PY{n}{nan}\PY{p}{)}
\PY{n}{array}\PY{p}{[}\PY{l+m+mi}{0}\PY{p}{,} \PY{p}{:}\PY{p}{]} \PY{o}{=} \PY{n}{np}\PY{o}{.}\PY{n}{random}\PY{o}{.}\PY{n}{random}\PY{p}{(}\PY{n}{ncols}\PY{p}{)}
\PY{n}{array}\PY{p}{[}\PY{p}{:}\PY{p}{,} \PY{l+m+mi}{0}\PY{p}{]} \PY{o}{=} \PY{n}{np}\PY{o}{.}\PY{n}{random}\PY{o}{.}\PY{n}{random}\PY{p}{(}\PY{n}{nrows}\PY{p}{)}
\PY{n}{array}\PY{p}{[}\PY{l+m+mi}{0}\PY{p}{:}\PY{l+m+mi}{6}\PY{p}{]}
\end{Verbatim}
\end{tcolorbox}

            \begin{tcolorbox}[breakable, size=fbox, boxrule=.5pt, pad at break*=1mm, opacityfill=0]
\prompt{Out}{outcolor}{ }{\boxspacing}
\begin{Verbatim}[commandchars=\\\{\}]
array([[0.61333107, 0.33702964, 0.80540485, 0.81565169, 0.49673407],
       [0.60278536,        nan,        nan,        nan,        nan],
       [0.66649055,        nan,        nan,        nan,        nan],
       [0.23308419,        nan,        nan,        nan,        nan],
       [0.56010969,        nan,        nan,        nan,        nan],
       [0.14735808,        nan,        nan,        nan,        nan]])
\end{Verbatim}
\end{tcolorbox}
        
    \begin{tcolorbox}[breakable, size=fbox, boxrule=1pt, pad at break*=1mm,colback=cellbackground, colframe=cellborder]
\prompt{In}{incolor}{ }{\boxspacing}
\begin{Verbatim}[commandchars=\\\{\}]
\PY{n}{p} \PY{o}{=} \PY{l+m+mf}{0.5}
\PY{n}{n} \PY{o}{=} \PY{n}{nrows}
\PY{n}{cols} \PY{o}{=} \PY{n}{np}\PY{o}{.}\PY{n}{random}\PY{o}{.}\PY{n}{geometric}\PY{p}{(}\PY{n}{p}\PY{p}{,} \PY{n}{n}\PY{p}{)}
\PY{n}{cols}\PY{p}{[}\PY{n}{cols} \PY{o}{\PYZgt{}}\PY{o}{=} \PY{n}{ncols}\PY{p}{]} \PY{o}{=} \PY{l+m+mi}{0}
\PY{n}{cols}
\end{Verbatim}
\end{tcolorbox}

            \begin{tcolorbox}[breakable, size=fbox, boxrule=.5pt, pad at break*=1mm, opacityfill=0]
\prompt{Out}{outcolor}{ }{\boxspacing}
\begin{Verbatim}[commandchars=\\\{\}]
array([2, 3, 4, 2, 2, 1, 2, 3, 0, 1, 1, 1, 1, 2, 1, 1, 4, 1, 2, 3, 2, 1,
       4, 1, 2, 2, 2, 1, 4, 1, 1, 1, 1, 1, 3, 2, 2, 3, 1, 1, 1, 2, 2, 1,
       4, 2, 1, 4, 1, 1, 1, 1, 1, 1, 1, 1, 1, 3, 3, 1, 1, 3, 1, 3, 2, 1,
       1, 4, 2, 3, 2, 1, 1, 1, 1, 1, 1, 1, 1, 2, 1, 1, 2, 1, 1, 1, 1, 1,
       1, 1, 4, 3, 1, 3, 1, 4, 2, 2, 1, 2])
\end{Verbatim}
\end{tcolorbox}
        
    \begin{tcolorbox}[breakable, size=fbox, boxrule=1pt, pad at break*=1mm,colback=cellbackground, colframe=cellborder]
\prompt{In}{incolor}{ }{\boxspacing}
\begin{Verbatim}[commandchars=\\\{\}]
\PY{n}{rows} \PY{o}{=} \PY{n}{np}\PY{o}{.}\PY{n}{random}\PY{o}{.}\PY{n}{randint}\PY{p}{(}\PY{n}{nrows}\PY{p}{,} \PY{n}{size}\PY{o}{=}\PY{n}{n}\PY{p}{)}
\PY{n}{rows}
\end{Verbatim}
\end{tcolorbox}

            \begin{tcolorbox}[breakable, size=fbox, boxrule=.5pt, pad at break*=1mm, opacityfill=0]
\prompt{Out}{outcolor}{ }{\boxspacing}
\begin{Verbatim}[commandchars=\\\{\}]
array([79, 51, 80, 44,  5, 12, 80, 42, 97, 14, 13, 45, 40, 76, 44, 28, 58,
       83, 93, 34, 26, 88, 34, 17, 24, 57, 87,  6, 45, 13, 91, 83, 28, 26,
        3, 79, 63, 31, 40,  8, 56, 31, 57, 50, 80, 39, 70, 72, 58, 71, 26,
       86,  6, 11,  1, 23, 57, 67, 65, 39, 78, 43, 22, 70,  3,  4, 55, 66,
       23, 93, 53, 54, 87, 33, 52, 72, 14, 64, 88, 69, 12,  3, 90, 48, 87,
       99, 34, 15, 33, 14, 95,  8, 57, 19, 41, 48, 24, 84,  9, 60])
\end{Verbatim}
\end{tcolorbox}
        
    \begin{tcolorbox}[breakable, size=fbox, boxrule=1pt, pad at break*=1mm,colback=cellbackground, colframe=cellborder]
\prompt{In}{incolor}{ }{\boxspacing}
\begin{Verbatim}[commandchars=\\\{\}]
\PY{n}{array}\PY{p}{[}\PY{n}{rows}\PY{p}{,} \PY{n}{cols}\PY{p}{]} \PY{o}{=} \PY{n}{np}\PY{o}{.}\PY{n}{random}\PY{o}{.}\PY{n}{random}\PY{p}{(}\PY{n}{n}\PY{p}{)}
\PY{n}{array}\PY{p}{[}\PY{l+m+mi}{0}\PY{p}{:}\PY{l+m+mi}{6}\PY{p}{]}
\end{Verbatim}
\end{tcolorbox}

            \begin{tcolorbox}[breakable, size=fbox, boxrule=.5pt, pad at break*=1mm, opacityfill=0]
\prompt{Out}{outcolor}{ }{\boxspacing}
\begin{Verbatim}[commandchars=\\\{\}]
array([[0.61333107, 0.33702964, 0.80540485, 0.81565169, 0.49673407],
       [0.60278536, 0.91261348,        nan,        nan,        nan],
       [0.66649055,        nan,        nan,        nan,        nan],
       [0.23308419, 0.42298515, 0.09855525, 0.65981556,        nan],
       [0.56010969, 0.32607767,        nan,        nan,        nan],
       [0.14735808,        nan, 0.25961923,        nan,        nan]])
\end{Verbatim}
\end{tcolorbox}
        
    \begin{tcolorbox}[breakable, size=fbox, boxrule=1pt, pad at break*=1mm,colback=cellbackground, colframe=cellborder]
\prompt{In}{incolor}{ }{\boxspacing}
\begin{Verbatim}[commandchars=\\\{\}]
\PY{n}{df} \PY{o}{=} \PY{n}{pd}\PY{o}{.}\PY{n}{DataFrame}\PY{p}{(}\PY{n}{array}\PY{p}{)}
\PY{n}{df}\PY{o}{.}\PY{n}{head}\PY{p}{(}\PY{p}{)}
\end{Verbatim}
\end{tcolorbox}

            \begin{tcolorbox}[breakable, size=fbox, boxrule=.5pt, pad at break*=1mm, opacityfill=0]
\prompt{Out}{outcolor}{ }{\boxspacing}
\begin{Verbatim}[commandchars=\\\{\}]
          0         1         2         3         4
0  0.613331  0.337030  0.805405  0.815652  0.496734
1  0.602785  0.912613       NaN       NaN       NaN
2  0.666491       NaN       NaN       NaN       NaN
3  0.233084  0.422985  0.098555  0.659816       NaN
4  0.560110  0.326078       NaN       NaN       NaN
\end{Verbatim}
\end{tcolorbox}
        
    \begin{tcolorbox}[breakable, size=fbox, boxrule=1pt, pad at break*=1mm,colback=cellbackground, colframe=cellborder]
\prompt{In}{incolor}{ }{\boxspacing}
\begin{Verbatim}[commandchars=\\\{\}]
\PY{n}{filled} \PY{o}{=} \PY{n}{df}\PY{o}{.}\PY{n}{fillna}\PY{p}{(}\PY{n}{method}\PY{o}{=}\PY{l+s+s1}{\PYZsq{}}\PY{l+s+s1}{ffill}\PY{l+s+s1}{\PYZsq{}}\PY{p}{,} \PY{n}{axis}\PY{o}{=}\PY{l+m+mi}{0}\PY{p}{)}
\PY{n}{filled}\PY{o}{.}\PY{n}{head}\PY{p}{(}\PY{p}{)}
\end{Verbatim}
\end{tcolorbox}

            \begin{tcolorbox}[breakable, size=fbox, boxrule=.5pt, pad at break*=1mm, opacityfill=0]
\prompt{Out}{outcolor}{ }{\boxspacing}
\begin{Verbatim}[commandchars=\\\{\}]
          0         1         2         3         4
0  0.613331  0.337030  0.805405  0.815652  0.496734
1  0.602785  0.912613  0.805405  0.815652  0.496734
2  0.666491  0.912613  0.805405  0.815652  0.496734
3  0.233084  0.422985  0.098555  0.659816  0.496734
4  0.560110  0.326078  0.098555  0.659816  0.496734
\end{Verbatim}
\end{tcolorbox}
        
    \begin{tcolorbox}[breakable, size=fbox, boxrule=1pt, pad at break*=1mm,colback=cellbackground, colframe=cellborder]
\prompt{In}{incolor}{ }{\boxspacing}
\begin{Verbatim}[commandchars=\\\{\}]
\PY{n}{total} \PY{o}{=} \PY{n}{filled}\PY{o}{.}\PY{n}{sum}\PY{p}{(}\PY{n}{axis}\PY{o}{=}\PY{l+m+mi}{1}\PY{p}{)}
\PY{n}{total}\PY{o}{.}\PY{n}{head}\PY{p}{(}\PY{p}{)}
\end{Verbatim}
\end{tcolorbox}

            \begin{tcolorbox}[breakable, size=fbox, boxrule=.5pt, pad at break*=1mm, opacityfill=0]
\prompt{Out}{outcolor}{ }{\boxspacing}
\begin{Verbatim}[commandchars=\\\{\}]
0    3.068151
1    3.633189
2    3.696895
3    1.911174
4    2.141292
dtype: float64
\end{Verbatim}
\end{tcolorbox}
        
    \begin{tcolorbox}[breakable, size=fbox, boxrule=1pt, pad at break*=1mm,colback=cellbackground, colframe=cellborder]
\prompt{In}{incolor}{ }{\boxspacing}
\begin{Verbatim}[commandchars=\\\{\}]
\PY{n}{wave} \PY{o}{=} \PY{n}{Wave}\PY{p}{(}\PY{n}{total}\PY{o}{.}\PY{n}{values}\PY{p}{)}
\PY{n}{wave}\PY{o}{.}\PY{n}{plot}\PY{p}{(}\PY{p}{)}
\end{Verbatim}
\end{tcolorbox}

    \begin{center}
    \adjustimage{max size={0.9\linewidth}{0.9\paperheight}}{Vasiliev_lab4_files/Vasiliev_lab4_50_0.png}
    \end{center}
    { \hspace*{\fill} \\}
    
    Реализуем алгоритм Voss-McCArtney.

    \begin{tcolorbox}[breakable, size=fbox, boxrule=1pt, pad at break*=1mm,colback=cellbackground, colframe=cellborder]
\prompt{In}{incolor}{ }{\boxspacing}
\begin{Verbatim}[commandchars=\\\{\}]
\PY{k}{def} \PY{n+nf}{voss}\PY{p}{(}\PY{n}{nrows}\PY{p}{,} \PY{n}{ncols}\PY{o}{=}\PY{l+m+mi}{16}\PY{p}{)}\PY{p}{:}
    \PY{n}{array} \PY{o}{=} \PY{n}{np}\PY{o}{.}\PY{n}{empty}\PY{p}{(}\PY{p}{(}\PY{n}{nrows}\PY{p}{,} \PY{n}{ncols}\PY{p}{)}\PY{p}{)}
    \PY{n}{array}\PY{o}{.}\PY{n}{fill}\PY{p}{(}\PY{n}{np}\PY{o}{.}\PY{n}{nan}\PY{p}{)}
    \PY{n}{array}\PY{p}{[}\PY{l+m+mi}{0}\PY{p}{,} \PY{p}{:}\PY{p}{]} \PY{o}{=} \PY{n}{np}\PY{o}{.}\PY{n}{random}\PY{o}{.}\PY{n}{random}\PY{p}{(}\PY{n}{ncols}\PY{p}{)}
    \PY{n}{array}\PY{p}{[}\PY{p}{:}\PY{p}{,} \PY{l+m+mi}{0}\PY{p}{]} \PY{o}{=} \PY{n}{np}\PY{o}{.}\PY{n}{random}\PY{o}{.}\PY{n}{random}\PY{p}{(}\PY{n}{nrows}\PY{p}{)}

    \PY{n}{n} \PY{o}{=} \PY{n}{nrows}
    \PY{n}{cols} \PY{o}{=} \PY{n}{np}\PY{o}{.}\PY{n}{random}\PY{o}{.}\PY{n}{geometric}\PY{p}{(}\PY{l+m+mf}{0.5}\PY{p}{,} \PY{n}{n}\PY{p}{)}
    \PY{n}{cols}\PY{p}{[}\PY{n}{cols} \PY{o}{\PYZgt{}}\PY{o}{=} \PY{n}{ncols}\PY{p}{]} \PY{o}{=} \PY{l+m+mi}{0}
    \PY{n}{rows} \PY{o}{=} \PY{n}{np}\PY{o}{.}\PY{n}{random}\PY{o}{.}\PY{n}{randint}\PY{p}{(}\PY{n}{nrows}\PY{p}{,} \PY{n}{size}\PY{o}{=}\PY{n}{n}\PY{p}{)}
    \PY{n}{array}\PY{p}{[}\PY{n}{rows}\PY{p}{,} \PY{n}{cols}\PY{p}{]} \PY{o}{=} \PY{n}{np}\PY{o}{.}\PY{n}{random}\PY{o}{.}\PY{n}{random}\PY{p}{(}\PY{n}{n}\PY{p}{)}

    \PY{n}{df} \PY{o}{=} \PY{n}{pd}\PY{o}{.}\PY{n}{DataFrame}\PY{p}{(}\PY{n}{array}\PY{p}{)}
    \PY{n}{df}\PY{o}{.}\PY{n}{fillna}\PY{p}{(}\PY{n}{method}\PY{o}{=}\PY{l+s+s1}{\PYZsq{}}\PY{l+s+s1}{ffill}\PY{l+s+s1}{\PYZsq{}}\PY{p}{,} \PY{n}{axis}\PY{o}{=}\PY{l+m+mi}{0}\PY{p}{,} \PY{n}{inplace}\PY{o}{=}\PY{k+kc}{True}\PY{p}{)}
    \PY{n}{total} \PY{o}{=} \PY{n}{df}\PY{o}{.}\PY{n}{sum}\PY{p}{(}\PY{n}{axis}\PY{o}{=}\PY{l+m+mi}{1}\PY{p}{)}

    \PY{k}{return} \PY{n}{total}\PY{o}{.}\PY{n}{values}
\end{Verbatim}
\end{tcolorbox}

    \begin{tcolorbox}[breakable, size=fbox, boxrule=1pt, pad at break*=1mm,colback=cellbackground, colframe=cellborder]
\prompt{In}{incolor}{ }{\boxspacing}
\begin{Verbatim}[commandchars=\\\{\}]
\PY{n}{ys} \PY{o}{=} \PY{n}{voss}\PY{p}{(}\PY{l+m+mi}{11025}\PY{p}{)}
\PY{n}{ys}
\end{Verbatim}
\end{tcolorbox}

            \begin{tcolorbox}[breakable, size=fbox, boxrule=.5pt, pad at break*=1mm, opacityfill=0]
\prompt{Out}{outcolor}{ }{\boxspacing}
\begin{Verbatim}[commandchars=\\\{\}]
array([6.12694777, 6.55530719, 6.49118506, {\ldots}, 8.63531383, 8.32892683,
       8.35285941])
\end{Verbatim}
\end{tcolorbox}
        
    \begin{tcolorbox}[breakable, size=fbox, boxrule=1pt, pad at break*=1mm,colback=cellbackground, colframe=cellborder]
\prompt{In}{incolor}{ }{\boxspacing}
\begin{Verbatim}[commandchars=\\\{\}]
\PY{n}{wave} \PY{o}{=} \PY{n}{Wave}\PY{p}{(}\PY{n}{ys}\PY{p}{)}
\PY{n}{wave}\PY{o}{.}\PY{n}{unbias}\PY{p}{(}\PY{p}{)}
\PY{n}{wave}\PY{o}{.}\PY{n}{normalize}\PY{p}{(}\PY{p}{)}
\end{Verbatim}
\end{tcolorbox}

    \begin{tcolorbox}[breakable, size=fbox, boxrule=1pt, pad at break*=1mm,colback=cellbackground, colframe=cellborder]
\prompt{In}{incolor}{ }{\boxspacing}
\begin{Verbatim}[commandchars=\\\{\}]
\PY{n}{wave}\PY{o}{.}\PY{n}{plot}\PY{p}{(}\PY{p}{)}
\end{Verbatim}
\end{tcolorbox}

    \begin{center}
    \adjustimage{max size={0.9\linewidth}{0.9\paperheight}}{Vasiliev_lab4_files/Vasiliev_lab4_55_0.png}
    \end{center}
    { \hspace*{\fill} \\}
    
    \begin{tcolorbox}[breakable, size=fbox, boxrule=1pt, pad at break*=1mm,colback=cellbackground, colframe=cellborder]
\prompt{In}{incolor}{ }{\boxspacing}
\begin{Verbatim}[commandchars=\\\{\}]
\PY{n}{wave}\PY{o}{.}\PY{n}{make\PYZus{}audio}\PY{p}{(}\PY{p}{)}
\end{Verbatim}
\end{tcolorbox}

            \begin{tcolorbox}[breakable, size=fbox, boxrule=.5pt, pad at break*=1mm, opacityfill=0]
\prompt{Out}{outcolor}{ }{\boxspacing}
\begin{Verbatim}[commandchars=\\\{\}]
<IPython.lib.display.Audio object>
\end{Verbatim}
\end{tcolorbox}
        
    \begin{tcolorbox}[breakable, size=fbox, boxrule=1pt, pad at break*=1mm,colback=cellbackground, colframe=cellborder]
\prompt{In}{incolor}{ }{\boxspacing}
\begin{Verbatim}[commandchars=\\\{\}]
\PY{n}{spectrum} \PY{o}{=} \PY{n}{wave}\PY{o}{.}\PY{n}{make\PYZus{}spectrum}\PY{p}{(}\PY{p}{)}
\PY{n}{spectrum}\PY{o}{.}\PY{n}{hs}\PY{p}{[}\PY{l+m+mi}{0}\PY{p}{]} \PY{o}{=} \PY{l+m+mi}{0}
\PY{n}{spectrum}\PY{o}{.}\PY{n}{plot\PYZus{}power}\PY{p}{(}\PY{p}{)}
\PY{n}{decorate}\PY{p}{(}\PY{n}{xlabel}\PY{o}{=}\PY{l+s+s1}{\PYZsq{}}\PY{l+s+s1}{Frequency (Hz)}\PY{l+s+s1}{\PYZsq{}}\PY{p}{,}
         \PY{n}{ylabel}\PY{o}{=}\PY{l+s+s1}{\PYZsq{}}\PY{l+s+s1}{Power}\PY{l+s+s1}{\PYZsq{}}\PY{p}{,}
         \PY{o}{*}\PY{o}{*}\PY{n}{loglog}\PY{p}{)}
\end{Verbatim}
\end{tcolorbox}

    \begin{center}
    \adjustimage{max size={0.9\linewidth}{0.9\paperheight}}{Vasiliev_lab4_files/Vasiliev_lab4_57_0.png}
    \end{center}
    { \hspace*{\fill} \\}
    
    \begin{tcolorbox}[breakable, size=fbox, boxrule=1pt, pad at break*=1mm,colback=cellbackground, colframe=cellborder]
\prompt{In}{incolor}{ }{\boxspacing}
\begin{Verbatim}[commandchars=\\\{\}]
\PY{n}{spectrum}\PY{o}{.}\PY{n}{estimate\PYZus{}slope}\PY{p}{(}\PY{p}{)}\PY{o}{.}\PY{n}{slope}
\end{Verbatim}
\end{tcolorbox}

            \begin{tcolorbox}[breakable, size=fbox, boxrule=.5pt, pad at break*=1mm, opacityfill=0]
\prompt{Out}{outcolor}{ }{\boxspacing}
\begin{Verbatim}[commandchars=\\\{\}]
-0.9707955385280218
\end{Verbatim}
\end{tcolorbox}
        
    \begin{tcolorbox}[breakable, size=fbox, boxrule=1pt, pad at break*=1mm,colback=cellbackground, colframe=cellborder]
\prompt{In}{incolor}{ }{\boxspacing}
\begin{Verbatim}[commandchars=\\\{\}]
\PY{n}{seg\PYZus{}length} \PY{o}{=} \PY{l+m+mi}{64} \PY{o}{*} \PY{l+m+mi}{1024}
\PY{n}{iters} \PY{o}{=} \PY{l+m+mi}{100}
\PY{n}{wave} \PY{o}{=} \PY{n}{Wave}\PY{p}{(}\PY{n}{voss}\PY{p}{(}\PY{n}{seg\PYZus{}length} \PY{o}{*} \PY{n}{iters}\PY{p}{)}\PY{p}{)}
\PY{n+nb}{len}\PY{p}{(}\PY{n}{wave}\PY{p}{)}
\end{Verbatim}
\end{tcolorbox}

            \begin{tcolorbox}[breakable, size=fbox, boxrule=.5pt, pad at break*=1mm, opacityfill=0]
\prompt{Out}{outcolor}{ }{\boxspacing}
\begin{Verbatim}[commandchars=\\\{\}]
6553600
\end{Verbatim}
\end{tcolorbox}
        
    \begin{tcolorbox}[breakable, size=fbox, boxrule=1pt, pad at break*=1mm,colback=cellbackground, colframe=cellborder]
\prompt{In}{incolor}{ }{\boxspacing}
\begin{Verbatim}[commandchars=\\\{\}]
\PY{n}{spectrum} \PY{o}{=} \PY{n}{bartlett\PYZus{}method}\PY{p}{(}\PY{n}{wave}\PY{p}{,} \PY{n}{seg\PYZus{}length}\PY{o}{=}\PY{n}{seg\PYZus{}length}\PY{p}{,} \PY{n}{win\PYZus{}flag}\PY{o}{=}\PY{k+kc}{False}\PY{p}{)}
\PY{n}{spectrum}\PY{o}{.}\PY{n}{hs}\PY{p}{[}\PY{l+m+mi}{0}\PY{p}{]} \PY{o}{=} \PY{l+m+mi}{0}
\PY{n+nb}{len}\PY{p}{(}\PY{n}{spectrum}\PY{p}{)}
\end{Verbatim}
\end{tcolorbox}

            \begin{tcolorbox}[breakable, size=fbox, boxrule=.5pt, pad at break*=1mm, opacityfill=0]
\prompt{Out}{outcolor}{ }{\boxspacing}
\begin{Verbatim}[commandchars=\\\{\}]
32769
\end{Verbatim}
\end{tcolorbox}
        
    \begin{tcolorbox}[breakable, size=fbox, boxrule=1pt, pad at break*=1mm,colback=cellbackground, colframe=cellborder]
\prompt{In}{incolor}{ }{\boxspacing}
\begin{Verbatim}[commandchars=\\\{\}]
\PY{n}{spectrum}\PY{o}{.}\PY{n}{plot\PYZus{}power}\PY{p}{(}\PY{p}{)}
\PY{n}{decorate}\PY{p}{(}\PY{n}{xlabel}\PY{o}{=}\PY{l+s+s1}{\PYZsq{}}\PY{l+s+s1}{Frequency (Hz)}\PY{l+s+s1}{\PYZsq{}}\PY{p}{,}
         \PY{n}{ylabel}\PY{o}{=}\PY{l+s+s1}{\PYZsq{}}\PY{l+s+s1}{Power}\PY{l+s+s1}{\PYZsq{}}\PY{p}{,}
         \PY{o}{*}\PY{o}{*}\PY{n}{loglog}\PY{p}{)}
\end{Verbatim}
\end{tcolorbox}

    \begin{center}
    \adjustimage{max size={0.9\linewidth}{0.9\paperheight}}{Vasiliev_lab4_files/Vasiliev_lab4_61_0.png}
    \end{center}
    { \hspace*{\fill} \\}
    
    \begin{tcolorbox}[breakable, size=fbox, boxrule=1pt, pad at break*=1mm,colback=cellbackground, colframe=cellborder]
\prompt{In}{incolor}{ }{\boxspacing}
\begin{Verbatim}[commandchars=\\\{\}]
\PY{n}{spectrum}\PY{o}{.}\PY{n}{estimate\PYZus{}slope}\PY{p}{(}\PY{p}{)}\PY{o}{.}\PY{n}{slope}
\end{Verbatim}
\end{tcolorbox}

            \begin{tcolorbox}[breakable, size=fbox, boxrule=.5pt, pad at break*=1mm, opacityfill=0]
\prompt{Out}{outcolor}{ }{\boxspacing}
\begin{Verbatim}[commandchars=\\\{\}]
-1.0020613235890785
\end{Verbatim}
\end{tcolorbox}
        

    % Add a bibliography block to the postdoc
    
    
    
\end{document}
